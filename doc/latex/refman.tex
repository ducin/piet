\documentclass[a4paper]{book}
\usepackage{a4wide}
\usepackage{makeidx}
\usepackage{fancyhdr}
\usepackage{graphicx}
\usepackage{multicol}
\usepackage{float}
\usepackage{textcomp}
\usepackage{alltt}
\usepackage{times}
\usepackage{ifpdf}
\ifpdf
\usepackage[pdftex,
            pagebackref=true,
            colorlinks=true,
            linkcolor=blue,
            unicode
           ]{hyperref}
\else
\usepackage[ps2pdf,
            pagebackref=true,
            colorlinks=true,
            linkcolor=blue,
            unicode
           ]{hyperref}
\usepackage{pspicture}
\fi
\usepackage[utf8]{inputenc}
\usepackage{polski}
\usepackage[T1]{fontenc}

\usepackage{doxygen}
\makeindex
\setcounter{tocdepth}{3}
\renewcommand{\footrulewidth}{0.4pt}
\begin{document}
\begin{titlepage}
\vspace*{7cm}
\begin{center}
{\Large Projekt Piet \\[1ex]\large wersja 1.0 }\\
\vspace*{1cm}
{\large Wygenerowano przez Doxygen 1.5.8}\\
\vspace*{0.5cm}
{\small Thu Sep 17 01:48:32 2009}\\
\end{center}
\end{titlepage}
\clearemptydoublepage
\pagenumbering{roman}
\tableofcontents
\clearemptydoublepage
\pagenumbering{arabic}
\chapter{Dokumentacja kodu projektu Piet}
\label{index}\hypertarget{index}{}{\bf Piet} to graficzny język programowania autorstwa {\bf Davida Morgan-Mar}. {\bf Interpreter} został stworzony przez {\bf Tomasza Ducina}. Poniższy tekst pochodzi ze wstępu do rozdziału 1, {\em \char`\"{}Język Piet\char`\"{}\/}, pracy magisterskiej Tomasza Ducina pt. {\em \char`\"{}Języki ezoteryczne Piet i Salvador jako uniwersalne maszyny obliczeniowe\char`\"{}\/}.

Język zawdzięcza swoją nazwę imieniu malarza, Pieta Mondriana, malarza który zapoczątkował gałąź geometrycznego malarstwa abstrakcyjnego. Z założenia, programy mają przypominać obrazy malarstwa abstrakcyjnego.

Interpreter Pieta wyposażony jest w głowicę, która odczytuje piksle z obrazu, przesuwając się po nim. Dane przetwarzane w trakcie działania programu są umieszczane na stosie.

Sposób pisania kodu (a szczególnie algorytmów obliczeniowych) różni się bardzo znacząco od programowania w innych językach nie tylko z powodu iż Piet jest językiem graficznym. Najważniejszymi dwiema cechami egzekucji kodu są: \mbox{[}1\mbox{]} porównywanie sąsiednich kolorów w celu wyboru instrukcji do wykonania (za pomocą głowicy) oraz \mbox{[}2\mbox{]} tzw. bloki kolorów (w których sąsiadujące piksle mają ten sam kolor). Na kod Pieta należy patrzeć „całościowo”, ogarniając kształt całości obrazu, a nie tylko dowolne niewielkie fragmenty. To jest właśnie zasadnicza różnica w sposobie programowania między Pietem i językami imperatywnymi (a również Salvadorem). W uproszczeniu, egzekucja kodu imperatywnego może wyglądać tak: „podstaw wartość pod zmienną, następnie na jej podstawie oblicz kolejne wyrażenie i podstaw pod inną zmienną. Teraz, mając przygotowane wszystkie pomocnicze wartości, sprawdzamy jedną z nich: jeśli sprawdzana wartość jest dodatnia, wykonaj blok instrukcji nr 1, jeśli ujemna, wykonaj blok nr 2, jeśli zero, wyświetl komunikat”. W językach imperatywnych mamy ściśle określoną, liniową kolejność wykonywania instrukcji, łatwo nam ją kontrolować. W Piecie natomiast kierunek w którym przesunie się głowica (czyli kolejność wykonywania instrukcji) zależy od trzech czynników: \mbox{[}1\mbox{]} aktualnego kierunku ruchu głowicy, \mbox{[}2\mbox{]} kolorów po których porusza się głowica i \mbox{[}3\mbox{]} kształtu bloków kolorów. Sprawia to, że programowanie skomplikowanych algorytmów jest stosunkowo trudne. 
\chapter{Indeks klas}
\section{Lista klas}
Tutaj znajdują się klasy, struktury, unie i interfejsy wraz z ich krótkimi opisami:\begin{CompactList}
\item\contentsline{section}{\hyperlink{classPBlockManager}{PBlockManager} (Manager bloków kolorów )}{\pageref{classPBlockManager}}{}
\item\contentsline{section}{\hyperlink{classPCalcStack}{PCalcStack} (Stos )}{\pageref{classPCalcStack}}{}
\item\contentsline{section}{\hyperlink{classPCodePointer}{PCodePointer} (Głowica obrazu kodu )}{\pageref{classPCodePointer}}{}
\item\contentsline{section}{\hyperlink{classPColorManager}{PColorManager} (Maszyna kodu )}{\pageref{classPColorManager}}{}
\item\contentsline{section}{\hyperlink{classPConsole}{PConsole} (Konsola wejścia/wyjścia )}{\pageref{classPConsole}}{}
\item\contentsline{section}{\hyperlink{classPVirtualMachine}{PVirtualMachine} (Wirtualna maszyna Pieta )}{\pageref{classPVirtualMachine}}{}
\item\contentsline{section}{\hyperlink{structstruct__point}{struct\_\-point} }{\pageref{structstruct__point}}{}
\end{CompactList}

\chapter{Indeks plików}
\section{Lista plików}
Tutaj znajduje się lista wszystkich udokumentowanych plików z ich krótkimi opisami:\begin{CompactList}
\item\contentsline{section}{src/\hyperlink{debug_8h}{debug.h} (Plik nagłówkowy debuggera )}{\pageref{debug_8h}}{}
\item\contentsline{section}{src/\hyperlink{test_8cpp}{test.cpp} (Plik z kodem źródłowym aplikacji )}{\pageref{test_8cpp}}{}
\item\contentsline{section}{src/core/\hyperlink{pblockmanager_8cpp}{pblockmanager.cpp} (Plik z kodem źródłowym klasy \hyperlink{classPBlockManager}{PBlockManager} )}{\pageref{pblockmanager_8cpp}}{}
\item\contentsline{section}{src/core/\hyperlink{pblockmanager_8h}{pblockmanager.h} (Plik nagłówkowy klasy \hyperlink{classPBlockManager}{PBlockManager} )}{\pageref{pblockmanager_8h}}{}
\item\contentsline{section}{src/core/\hyperlink{pcalcstack_8cpp}{pcalcstack.cpp} (Plik z kodem źródłowym klasy \hyperlink{classPCalcStack}{PCalcStack} )}{\pageref{pcalcstack_8cpp}}{}
\item\contentsline{section}{src/core/\hyperlink{pcalcstack_8h}{pcalcstack.h} (Plik nagłówkowy klasy \hyperlink{classPCalcStack}{PCalcStack} )}{\pageref{pcalcstack_8h}}{}
\item\contentsline{section}{src/core/\hyperlink{pcodepointer_8cpp}{pcodepointer.cpp} (Plik z kodem źródłowym klasy \hyperlink{classPCodePointer}{PCodePointer} )}{\pageref{pcodepointer_8cpp}}{}
\item\contentsline{section}{src/core/\hyperlink{pcodepointer_8h}{pcodepointer.h} (Plik nagłówkowy klasy \hyperlink{classPCodePointer}{PCodePointer} )}{\pageref{pcodepointer_8h}}{}
\item\contentsline{section}{src/core/\hyperlink{pcolormanager_8cpp}{pcolormanager.cpp} (Plik z kodem źródłowym klasy \hyperlink{classPColorManager}{PColorManager} )}{\pageref{pcolormanager_8cpp}}{}
\item\contentsline{section}{src/core/\hyperlink{pcolormanager_8h}{pcolormanager.h} (Plik nagłówkowy klasy \hyperlink{classPColorManager}{PColorManager} )}{\pageref{pcolormanager_8h}}{}
\item\contentsline{section}{src/core/\hyperlink{pconsole_8cpp}{pconsole.cpp} (Plik z kodem źródłowym klasy \hyperlink{classPConsole}{PConsole} )}{\pageref{pconsole_8cpp}}{}
\item\contentsline{section}{src/core/\hyperlink{pconsole_8h}{pconsole.h} (Plik nagłówkowy klasy \hyperlink{classPConsole}{PConsole} )}{\pageref{pconsole_8h}}{}
\item\contentsline{section}{src/core/\hyperlink{penums_8h}{penums.h} (Plik nagłówkowy z definicjami enumeracji )}{\pageref{penums_8h}}{}
\item\contentsline{section}{src/core/\hyperlink{pstructs_8h}{pstructs.h} (Plik nagłówkowy z definicjami rekordów )}{\pageref{pstructs_8h}}{}
\item\contentsline{section}{src/core/\hyperlink{pvirtualmachine_8cpp}{pvirtualmachine.cpp} (Plik z kodem źródłowym klasy \hyperlink{classPVirtualMachine}{PVirtualMachine} )}{\pageref{pvirtualmachine_8cpp}}{}
\item\contentsline{section}{src/core/\hyperlink{pvirtualmachine_8h}{pvirtualmachine.h} (Plik nagłówkowy klasy \hyperlink{classPVirtualMachine}{PVirtualMachine} )}{\pageref{pvirtualmachine_8h}}{}
\end{CompactList}

\chapter{Dokumentacja klas}
\hypertarget{classPBlockManager}{
\section{Dokumentacja klasy PBlockManager}
\label{classPBlockManager}\index{PBlockManager@{PBlockManager}}
}
Manager bloków kolorów.  


{\tt \#include $<$pblockmanager.h$>$}

\subsection*{Metody publiczne}
\begin{CompactItemize}
\item 
\hyperlink{classPBlockManager_f5a95fca85a31c0257b38b57cbb3378a}{PBlockManager} (QImage $\ast$, \hyperlink{classPCodePointer}{PCodePointer} $\ast$)
\begin{CompactList}\small\item\em konstruktor klasy \hyperlink{classPBlockManager}{PBlockManager} \item\end{CompactList}\item 
\hyperlink{classPBlockManager_52d7abb7f858625e9e014fb555f571a6}{$\sim$PBlockManager} ()
\begin{CompactList}\small\item\em destruktor \item\end{CompactList}\item 
\hypertarget{classPBlockManager_926e182e360300d6524fbb01fb90866b}{
void \textbf{setVerbosity} (bool)}
\label{classPBlockManager_926e182e360300d6524fbb01fb90866b}

\item 
\hypertarget{classPBlockManager_88f9f5226179e0fbd9d148001de10e97}{
void \textbf{searchAndFillCodels} ()}
\label{classPBlockManager_88f9f5226179e0fbd9d148001de10e97}

\item 
\hypertarget{classPBlockManager_ba7919eedd6b806f2f4da408423723ea}{
int \textbf{getCodelBlockCount} ()}
\label{classPBlockManager_ba7919eedd6b806f2f4da408423723ea}

\item 
\hypertarget{classPBlockManager_667f3673eda6cf4476ecb34d01db665b}{
PPoint \textbf{getNextPossibleCodel} ()}
\label{classPBlockManager_667f3673eda6cf4476ecb34d01db665b}

\item 
\hypertarget{classPBlockManager_e6298891058bd316c07071db0c851b2f}{
void \textbf{\_\-\_\-dev\_\-\_\-showMultiArray} ()}
\label{classPBlockManager_e6298891058bd316c07071db0c851b2f}

\item 
\hypertarget{classPBlockManager_d0ed8ac41d59c34eb88aa30428961a82}{
void \textbf{\_\-\_\-dev\_\-\_\-showCountAndBorderCodels} ()}
\label{classPBlockManager_d0ed8ac41d59c34eb88aa30428961a82}

\end{CompactItemize}
\subsection*{Metody chronione}
\begin{CompactItemize}
\item 
\hypertarget{classPBlockManager_c31dc0993286be444a82ae5e51f32415}{
void \textbf{fillMultiArray} (int)}
\label{classPBlockManager_c31dc0993286be444a82ae5e51f32415}

\item 
\hypertarget{classPBlockManager_ab82616c7ee28fa12d4cfec944f550a0}{
void \textbf{clearMultiArray} ()}
\label{classPBlockManager_ab82616c7ee28fa12d4cfec944f550a0}

\end{CompactItemize}
\subsection*{Atrybuty chronione}
\begin{CompactItemize}
\item 
\hypertarget{classPBlockManager_429ddf44eba8dc97ccd38d67d4b8172f}{
QImage $\ast$ \textbf{image}}
\label{classPBlockManager_429ddf44eba8dc97ccd38d67d4b8172f}

\item 
\hypertarget{classPBlockManager_835bfe00bdbea928d1b4ad6212cf733c}{
\hyperlink{classPCodePointer}{PCodePointer} $\ast$ \textbf{pointer}}
\label{classPBlockManager_835bfe00bdbea928d1b4ad6212cf733c}

\item 
\hypertarget{classPBlockManager_679ed25c188f4a6d2fd20a167a4bb4eb}{
int \textbf{codel\_\-block\_\-count}}
\label{classPBlockManager_679ed25c188f4a6d2fd20a167a4bb4eb}

\item 
\hypertarget{classPBlockManager_8d73c339ef6be4cd3a960b96b0c4f1fe}{
int \textbf{border\_\-right\_\-codel}}
\label{classPBlockManager_8d73c339ef6be4cd3a960b96b0c4f1fe}

\item 
\hypertarget{classPBlockManager_9a2e17b1fa59c0b083e760a385b3565f}{
int \textbf{border\_\-down\_\-codel}}
\label{classPBlockManager_9a2e17b1fa59c0b083e760a385b3565f}

\item 
\hypertarget{classPBlockManager_2bb1887ef5658a211da5edbc3a248058}{
int \textbf{border\_\-left\_\-codel}}
\label{classPBlockManager_2bb1887ef5658a211da5edbc3a248058}

\item 
\hypertarget{classPBlockManager_80de22998ec78cc92162eadd6036506f}{
int \textbf{border\_\-up\_\-codel}}
\label{classPBlockManager_80de22998ec78cc92162eadd6036506f}

\end{CompactItemize}


\subsection{Opis szczegółowy}
Manager bloków kolorów. 

Klasa odpowiedzialna właściwie tylko za jedną, ale dosyć skomplikowaną operację - obliczanie ilości kodeli w danym bloku koloru. Ta wydzielona klasa posiada własne struktury danych i algorytm i zwraca maszynie kodu (klasie opakowującej) liczbę kodeli w danym bloku. 

\subsection{Dokumentacja konstruktora i destruktora}
\hypertarget{classPBlockManager_f5a95fca85a31c0257b38b57cbb3378a}{
\index{PBlockManager@{PBlockManager}!PBlockManager@{PBlockManager}}
\index{PBlockManager@{PBlockManager}!PBlockManager@{PBlockManager}}
\subsubsection[{PBlockManager}]{\setlength{\rightskip}{0pt plus 5cm}PBlockManager::PBlockManager (QImage $\ast$ {\em given\_\-image}, \/  {\bf PCodePointer} $\ast$ {\em given\_\-pointer})}}
\label{classPBlockManager_f5a95fca85a31c0257b38b57cbb3378a}


konstruktor klasy \hyperlink{classPBlockManager}{PBlockManager} 

xyz... \hypertarget{classPBlockManager_52d7abb7f858625e9e014fb555f571a6}{
\index{PBlockManager@{PBlockManager}!$\sim$PBlockManager@{$\sim$PBlockManager}}
\index{$\sim$PBlockManager@{$\sim$PBlockManager}!PBlockManager@{PBlockManager}}
\subsubsection[{$\sim$PBlockManager}]{\setlength{\rightskip}{0pt plus 5cm}PBlockManager::$\sim$PBlockManager ()}}
\label{classPBlockManager_52d7abb7f858625e9e014fb555f571a6}


destruktor 

abc... 

Dokumentacja dla tej klasy została wygenerowana z plików:\begin{CompactItemize}
\item 
src/core/\hyperlink{pblockmanager_8h}{pblockmanager.h}\item 
src/core/\hyperlink{pblockmanager_8cpp}{pblockmanager.cpp}\end{CompactItemize}

\hypertarget{classPCalcStack}{
\section{Dokumentacja klasy PCalcStack}
\label{classPCalcStack}\index{PCalcStack@{PCalcStack}}
}
Stos.  


{\tt \#include $<$pcalcstack.h$>$}

\subsection*{Metody publiczne}
\begin{CompactItemize}
\item 
\hypertarget{classPCalcStack_6a831a20f4dec457de9f880202cf2e33}{
void \textbf{setVerbosity} (bool)}
\label{classPCalcStack_6a831a20f4dec457de9f880202cf2e33}

\item 
\hypertarget{classPCalcStack_7b1abd7a77db664dcf9e0bd56c17301f}{
void \textbf{clear} ()}
\label{classPCalcStack_7b1abd7a77db664dcf9e0bd56c17301f}

\item 
\hypertarget{classPCalcStack_b3129921fea95a4df231201e2601e1c0}{
void \textbf{prepareToExecute} ()}
\label{classPCalcStack_b3129921fea95a4df231201e2601e1c0}

\item 
\hypertarget{classPCalcStack_e334a37c1749b028e75e0d251afe47f7}{
int \textbf{size} ()}
\label{classPCalcStack_e334a37c1749b028e75e0d251afe47f7}

\item 
\hypertarget{classPCalcStack_39d4d9e53e8b57fa934ee1a5b9171dc8}{
bool \textbf{hasAtLeastNElements} (unsigned int)}
\label{classPCalcStack_39d4d9e53e8b57fa934ee1a5b9171dc8}

\item 
\hypertarget{classPCalcStack_269ca0f8f0a4395a0035fa0a9bc813a1}{
void \textbf{instrPush} (int)}
\label{classPCalcStack_269ca0f8f0a4395a0035fa0a9bc813a1}

\item 
\hypertarget{classPCalcStack_796ba40ffc60b05a8a116d881aa2e37f}{
int \textbf{instrPop} ()}
\label{classPCalcStack_796ba40ffc60b05a8a116d881aa2e37f}

\item 
\hypertarget{classPCalcStack_04ef30536acac1b02093153f1cb83353}{
void \textbf{instrAdd} ()}
\label{classPCalcStack_04ef30536acac1b02093153f1cb83353}

\item 
\hypertarget{classPCalcStack_a35493e8dc971fc2f47a113a6d400784}{
void \textbf{instrSubtract} ()}
\label{classPCalcStack_a35493e8dc971fc2f47a113a6d400784}

\item 
\hypertarget{classPCalcStack_0d2b0b03f9a14c875364d4a5dabb6d7a}{
void \textbf{instrMultiply} ()}
\label{classPCalcStack_0d2b0b03f9a14c875364d4a5dabb6d7a}

\item 
\hypertarget{classPCalcStack_f642cacff51d21c2f25487dae1a5b84c}{
void \textbf{instrDivide} ()}
\label{classPCalcStack_f642cacff51d21c2f25487dae1a5b84c}

\item 
\hypertarget{classPCalcStack_bda10d125ac0b2dee9398ce185435fdb}{
void \textbf{instrModulo} ()}
\label{classPCalcStack_bda10d125ac0b2dee9398ce185435fdb}

\item 
\hypertarget{classPCalcStack_5fe0eb9337f3e18416e02c082c14dc66}{
void \textbf{instrNot} ()}
\label{classPCalcStack_5fe0eb9337f3e18416e02c082c14dc66}

\item 
\hypertarget{classPCalcStack_b656cedeb6038c6e3af6e01df21d8c1b}{
void \textbf{instrGreater} ()}
\label{classPCalcStack_b656cedeb6038c6e3af6e01df21d8c1b}

\item 
\hypertarget{classPCalcStack_0424a6bcacc8bee1a5ff074dea87be1e}{
void \textbf{instrDuplicate} ()}
\label{classPCalcStack_0424a6bcacc8bee1a5ff074dea87be1e}

\item 
\hypertarget{classPCalcStack_1be80d765fdc530467d76594d94c6746}{
void \textbf{instrRoll} ()}
\label{classPCalcStack_1be80d765fdc530467d76594d94c6746}

\item 
\hypertarget{classPCalcStack_e28932817d21f528cb3ebe06471df9a2}{
void \textbf{\_\-\_\-dev\_\-\_\-printAllStackValues} ()}
\label{classPCalcStack_e28932817d21f528cb3ebe06471df9a2}

\item 
\hypertarget{classPCalcStack_dbfaddfaadb912324c965390a70ae323}{
void \textbf{\_\-\_\-dev\_\-\_\-printConsole} ()}
\label{classPCalcStack_dbfaddfaadb912324c965390a70ae323}

\end{CompactItemize}
\subsection*{Atrybuty chronione}
\begin{CompactItemize}
\item 
\hypertarget{classPCalcStack_4777b5f859158a71a18deb0bc0978653}{
std::list$<$ int $>$ \textbf{values}}
\label{classPCalcStack_4777b5f859158a71a18deb0bc0978653}

\end{CompactItemize}


\subsection{Opis szczegółowy}
Stos. 

Klasa pełniąca funkcję stosu wykorzystywanego do interpretowania kodu Pieta. Wszystkie jej funkcjonalności są dokładnie określone przez instrukcje dozwolone w języku Piet - począwszy od podstawowych (kładzenie/zdejmowanie elementów), poprzez arytmetyczne i logiczne (zmieniające elementy na szczycie stosu w zależności od ich wartości) a skończywszy na operacjach sterujących głowicą (przestawianie 'codel chooser' i 'direction pointer' - patrz: specyfikacja Piet), operacjach modyfikujących strukturę elementów stosu oraz operacjach I/O.

Stanowi jeden z dwóch obiektów potrzebnych do pełnej interpretacji kodu, używanych przez tzw. \char`\"{}wirtualną maszynę Pieta\char`\"{}, najwyższy w hierarchi obiekt. 

Dokumentacja dla tej klasy została wygenerowana z plików:\begin{CompactItemize}
\item 
src/core/\hyperlink{pcalcstack_8h}{pcalcstack.h}\item 
src/core/\hyperlink{pcalcstack_8cpp}{pcalcstack.cpp}\end{CompactItemize}

\hypertarget{classPCodePointer}{
\section{Dokumentacja klasy PCodePointer}
\label{classPCodePointer}\index{PCodePointer@{PCodePointer}}
}
Głowica obrazu kodu.  


{\tt \#include $<$pcodepointer.h$>$}

\subsection*{Metody publiczne}
\begin{CompactItemize}
\item 
\hyperlink{classPCodePointer_cd5cd1cd86ff9cf89ac6d2aae5cb52d8}{PCodePointer} (QImage $\ast$, PPoint)
\item 
\hyperlink{classPCodePointer_dc12b4bda6ed4b1045b0767b41ff2efa}{$\sim$PCodePointer} ()
\item 
void \hyperlink{classPCodePointer_d738019c8cb766c0863e2c00622d9fc9}{setVerbosity} (bool)
\item 
\hypertarget{classPCodePointer_fb47282905de53ad98a0c3d76dbd0fb0}{
void \textbf{clear} ()}
\label{classPCodePointer_fb47282905de53ad98a0c3d76dbd0fb0}

\item 
PPoint \hyperlink{classPCodePointer_c792e5bc527542482542ed22acc9cca4}{getCoordinates} ()
\item 
void \hyperlink{classPCodePointer_9bcf58c97f704e3be2deefb588f91312}{setCoordinateX} (int)
\item 
void \hyperlink{classPCodePointer_ef8324dbdca82baa094e47e36f978669}{incCoordinateX} ()
\item 
void \hyperlink{classPCodePointer_3b98e9637236aa9d975dc59c0397e625}{decCoordinateX} ()
\item 
void \hyperlink{classPCodePointer_9fd77f14e39cc30f8cfc583ce97aa2f2}{setCoordinateY} (int)
\item 
void \hyperlink{classPCodePointer_58c15d4d1abb1fb971170f19c3b9e8cf}{incCoordinateY} ()
\item 
void \hyperlink{classPCodePointer_cc8b95bb3786748c1aff41eed6d299e1}{decCoordinateY} ()
\item 
void \hyperlink{classPCodePointer_117bf9322b4fdcb189c41a6f2f113a4b}{setCoordinates} (PPoint)
\item 
\hypertarget{classPCodePointer_6544ee43112dcafb5fc7c865ced682f0}{
QRgb \textbf{getPointedPixel} ()}
\label{classPCodePointer_6544ee43112dcafb5fc7c865ced682f0}

\item 
\hypertarget{classPCodePointer_e9f3396a98b84d090e820ad37c6dca21}{
QRgb \textbf{getPixel} (PPoint)}
\label{classPCodePointer_e9f3396a98b84d090e820ad37c6dca21}

\item 
bool \hyperlink{classPCodePointer_a9e836354b61f96ed39208c8b37d57d3}{pointOutsideImage} (PPoint)
\item 
\hyperlink{penums_8h_6d3256570150238c718cbbb5f81c82df}{PDirectionPointerValues} \hyperlink{classPCodePointer_e0461a3d72af876b5b3c838a8b45a729}{getDirectionPointerValue} ()
\item 
\hyperlink{penums_8h_59dc57d526e2ce263bdf851c0d4fef3e}{PCodelChooserValues} \hyperlink{classPCodePointer_529b0f2f65c1d17082d755813fec0194}{getCodelChooserValue} ()
\item 
void \hyperlink{classPCodePointer_56d2ef632779fbe64937030e82b027d1}{toggleCodelChooser} ()
\item 
void \hyperlink{classPCodePointer_4aab1f30e01bb0fb3d78b5e6aa93535c}{toggleDirectionPointer} ()
\item 
void \hyperlink{classPCodePointer_5b34ab0f6bb3ddb3bd7e44a7d9a613ca}{toggle} ()
\item 
void \hyperlink{classPCodePointer_ce0986ca3dc000a06a0155d20cd83d26}{\_\-\_\-dev\_\-\_\-printCoordinates} ()
\item 
void \hyperlink{classPCodePointer_103c79ebd3257261d5c72322bc4eb742}{\_\-\_\-dev\_\-\_\-printDirectionPointer} ()
\item 
void \hyperlink{classPCodePointer_21810c2cdb66ea83cabafde33e8a0206}{\_\-\_\-dev\_\-\_\-printCodelChooser} ()
\item 
void \hyperlink{classPCodePointer_fe765c0487ab59dbbbdd47b78842bbcb}{\_\-\_\-dev\_\-\_\-printConsole} ()
\end{CompactItemize}
\subsection*{Metody chronione}
\begin{CompactItemize}
\item 
void \hyperlink{classPCodePointer_b5c13bf294dddd672ca9fe7675eb258b}{setCodelChooser} (\hyperlink{penums_8h_59dc57d526e2ce263bdf851c0d4fef3e}{PCodelChooserValues})
\item 
void \hyperlink{classPCodePointer_0b97bd6b4383e976e54c5738178c7815}{setDirectionPointer} (\hyperlink{penums_8h_6d3256570150238c718cbbb5f81c82df}{PDirectionPointerValues})
\item 
void \hyperlink{classPCodePointer_a3ad29e6327d54faf0b081892e720aca}{turnDirectionPointerClockwise} ()
\item 
void \hyperlink{classPCodePointer_3f43e8205a0554bf7baa3821e01ab4a2}{turnDirectionPointerAnticlockwise} ()
\end{CompactItemize}
\subsection*{Atrybuty chronione}
\begin{CompactItemize}
\item 
QImage $\ast$ \hyperlink{classPCodePointer_9f6689e6425046abd85001f915ba1221}{image}
\item 
\hyperlink{penums_8h_59dc57d526e2ce263bdf851c0d4fef3e}{PCodelChooserValues} \hyperlink{classPCodePointer_a3d7ea2563c9e1363ddcb0fb19b35f5a}{codel\_\-chooser}
\item 
\hyperlink{penums_8h_6d3256570150238c718cbbb5f81c82df}{PDirectionPointerValues} \hyperlink{classPCodePointer_b285be0011b9b5c8b35bd1ba27ee935f}{direction\_\-pointer}
\end{CompactItemize}


\subsection{Opis szczegółowy}
Głowica obrazu kodu. 

Klasa pełniąca funkcję głowicy która porusza się po tzw. \char`\"{}obrazie kodu\char`\"{}. Wykonuje wszystkie operacje bezpośrednio związane z odczytywaniem piksli z obrazu, posiada (zmieniające się w trakcie działania programu) współrzędne oraz dodatkowe elementy, takie jak 'codel chooser' i 'direction pointer' (patrz: specyfikacja języka Piet).

Obok managera kolorów i managera bloków koloru jest trzecim elementem wykorzystywanym przez tzw. \char`\"{}maszynę kodu\char`\"{} i (razem z nimi) służy do właściwej, jednoznacznej interpretacji kodu Pieta. 

\subsection{Dokumentacja konstruktora i destruktora}
\hypertarget{classPCodePointer_cd5cd1cd86ff9cf89ac6d2aae5cb52d8}{
\index{PCodePointer@{PCodePointer}!PCodePointer@{PCodePointer}}
\index{PCodePointer@{PCodePointer}!PCodePointer@{PCodePointer}}
\subsubsection[{PCodePointer}]{\setlength{\rightskip}{0pt plus 5cm}PCodePointer::PCodePointer (QImage $\ast$ {\em code\_\-image}, \/  PPoint {\em initial})}}
\label{classPCodePointer_cd5cd1cd86ff9cf89ac6d2aae5cb52d8}


Konstruktor głowicy obrazu kodu. Nie robi nic szczególnego. \hypertarget{classPCodePointer_dc12b4bda6ed4b1045b0767b41ff2efa}{
\index{PCodePointer@{PCodePointer}!$\sim$PCodePointer@{$\sim$PCodePointer}}
\index{$\sim$PCodePointer@{$\sim$PCodePointer}!PCodePointer@{PCodePointer}}
\subsubsection[{$\sim$PCodePointer}]{\setlength{\rightskip}{0pt plus 5cm}PCodePointer::$\sim$PCodePointer ()}}
\label{classPCodePointer_dc12b4bda6ed4b1045b0767b41ff2efa}


Destruktor głowicy obrazu kodu. Nie robi nic szczególnego. 

\subsection{Dokumentacja funkcji składowych}
\hypertarget{classPCodePointer_21810c2cdb66ea83cabafde33e8a0206}{
\index{PCodePointer@{PCodePointer}!\_\-\_\-dev\_\-\_\-printCodelChooser@{\_\-\_\-dev\_\-\_\-printCodelChooser}}
\index{\_\-\_\-dev\_\-\_\-printCodelChooser@{\_\-\_\-dev\_\-\_\-printCodelChooser}!PCodePointer@{PCodePointer}}
\subsubsection[{\_\-\_\-dev\_\-\_\-printCodelChooser}]{\setlength{\rightskip}{0pt plus 5cm}void PCodePointer::\_\-\_\-dev\_\-\_\-printCodelChooser ()}}
\label{classPCodePointer_21810c2cdb66ea83cabafde33e8a0206}


METODA TESTOWA. Wyświetla na konsoli informacje o CC. \hypertarget{classPCodePointer_fe765c0487ab59dbbbdd47b78842bbcb}{
\index{PCodePointer@{PCodePointer}!\_\-\_\-dev\_\-\_\-printConsole@{\_\-\_\-dev\_\-\_\-printConsole}}
\index{\_\-\_\-dev\_\-\_\-printConsole@{\_\-\_\-dev\_\-\_\-printConsole}!PCodePointer@{PCodePointer}}
\subsubsection[{\_\-\_\-dev\_\-\_\-printConsole}]{\setlength{\rightskip}{0pt plus 5cm}void PCodePointer::\_\-\_\-dev\_\-\_\-printConsole ()}}
\label{classPCodePointer_fe765c0487ab59dbbbdd47b78842bbcb}


METODA TESTOWA. Wyświetla na konsoli wszystkie informacje o głowicy. \hypertarget{classPCodePointer_ce0986ca3dc000a06a0155d20cd83d26}{
\index{PCodePointer@{PCodePointer}!\_\-\_\-dev\_\-\_\-printCoordinates@{\_\-\_\-dev\_\-\_\-printCoordinates}}
\index{\_\-\_\-dev\_\-\_\-printCoordinates@{\_\-\_\-dev\_\-\_\-printCoordinates}!PCodePointer@{PCodePointer}}
\subsubsection[{\_\-\_\-dev\_\-\_\-printCoordinates}]{\setlength{\rightskip}{0pt plus 5cm}void PCodePointer::\_\-\_\-dev\_\-\_\-printCoordinates ()}}
\label{classPCodePointer_ce0986ca3dc000a06a0155d20cd83d26}


METODA TESTOWA. Wyświetla na konsoli informacje o współrzędnych głowicy. \hypertarget{classPCodePointer_103c79ebd3257261d5c72322bc4eb742}{
\index{PCodePointer@{PCodePointer}!\_\-\_\-dev\_\-\_\-printDirectionPointer@{\_\-\_\-dev\_\-\_\-printDirectionPointer}}
\index{\_\-\_\-dev\_\-\_\-printDirectionPointer@{\_\-\_\-dev\_\-\_\-printDirectionPointer}!PCodePointer@{PCodePointer}}
\subsubsection[{\_\-\_\-dev\_\-\_\-printDirectionPointer}]{\setlength{\rightskip}{0pt plus 5cm}void PCodePointer::\_\-\_\-dev\_\-\_\-printDirectionPointer ()}}
\label{classPCodePointer_103c79ebd3257261d5c72322bc4eb742}


METODA TESTOWA. Wyświetla na konsoli informacje o DP. \hypertarget{classPCodePointer_3b98e9637236aa9d975dc59c0397e625}{
\index{PCodePointer@{PCodePointer}!decCoordinateX@{decCoordinateX}}
\index{decCoordinateX@{decCoordinateX}!PCodePointer@{PCodePointer}}
\subsubsection[{decCoordinateX}]{\setlength{\rightskip}{0pt plus 5cm}void PCodePointer::decCoordinateX ()}}
\label{classPCodePointer_3b98e9637236aa9d975dc59c0397e625}


Dekrementuje współrzędną X głowicy. \hypertarget{classPCodePointer_cc8b95bb3786748c1aff41eed6d299e1}{
\index{PCodePointer@{PCodePointer}!decCoordinateY@{decCoordinateY}}
\index{decCoordinateY@{decCoordinateY}!PCodePointer@{PCodePointer}}
\subsubsection[{decCoordinateY}]{\setlength{\rightskip}{0pt plus 5cm}void PCodePointer::decCoordinateY ()}}
\label{classPCodePointer_cc8b95bb3786748c1aff41eed6d299e1}


Dekrementuje współrzędną Y głowicy. \hypertarget{classPCodePointer_529b0f2f65c1d17082d755813fec0194}{
\index{PCodePointer@{PCodePointer}!getCodelChooserValue@{getCodelChooserValue}}
\index{getCodelChooserValue@{getCodelChooserValue}!PCodePointer@{PCodePointer}}
\subsubsection[{getCodelChooserValue}]{\setlength{\rightskip}{0pt plus 5cm}{\bf PCodelChooserValues} PCodePointer::getCodelChooserValue ()}}
\label{classPCodePointer_529b0f2f65c1d17082d755813fec0194}


Zwraca wartość DP. \begin{Desc}
\item[Zwraca:]wartość DP \end{Desc}
\hypertarget{classPCodePointer_c792e5bc527542482542ed22acc9cca4}{
\index{PCodePointer@{PCodePointer}!getCoordinates@{getCoordinates}}
\index{getCoordinates@{getCoordinates}!PCodePointer@{PCodePointer}}
\subsubsection[{getCoordinates}]{\setlength{\rightskip}{0pt plus 5cm}PPoint PCodePointer::getCoordinates ()}}
\label{classPCodePointer_c792e5bc527542482542ed22acc9cca4}


Zwraca współrzędne głowicy. \begin{Desc}
\item[Zwraca:]współrzędne głowicy \end{Desc}
\hypertarget{classPCodePointer_e0461a3d72af876b5b3c838a8b45a729}{
\index{PCodePointer@{PCodePointer}!getDirectionPointerValue@{getDirectionPointerValue}}
\index{getDirectionPointerValue@{getDirectionPointerValue}!PCodePointer@{PCodePointer}}
\subsubsection[{getDirectionPointerValue}]{\setlength{\rightskip}{0pt plus 5cm}{\bf PDirectionPointerValues} PCodePointer::getDirectionPointerValue ()}}
\label{classPCodePointer_e0461a3d72af876b5b3c838a8b45a729}


Zwraca wartość DP. \begin{Desc}
\item[Zwraca:]wartość DP \end{Desc}
\hypertarget{classPCodePointer_ef8324dbdca82baa094e47e36f978669}{
\index{PCodePointer@{PCodePointer}!incCoordinateX@{incCoordinateX}}
\index{incCoordinateX@{incCoordinateX}!PCodePointer@{PCodePointer}}
\subsubsection[{incCoordinateX}]{\setlength{\rightskip}{0pt plus 5cm}void PCodePointer::incCoordinateX ()}}
\label{classPCodePointer_ef8324dbdca82baa094e47e36f978669}


Inkrementuje współrzędną X głowicy. \hypertarget{classPCodePointer_58c15d4d1abb1fb971170f19c3b9e8cf}{
\index{PCodePointer@{PCodePointer}!incCoordinateY@{incCoordinateY}}
\index{incCoordinateY@{incCoordinateY}!PCodePointer@{PCodePointer}}
\subsubsection[{incCoordinateY}]{\setlength{\rightskip}{0pt plus 5cm}void PCodePointer::incCoordinateY ()}}
\label{classPCodePointer_58c15d4d1abb1fb971170f19c3b9e8cf}


Inkrementuje współrzędną Y głowicy. \hypertarget{classPCodePointer_a9e836354b61f96ed39208c8b37d57d3}{
\index{PCodePointer@{PCodePointer}!pointOutsideImage@{pointOutsideImage}}
\index{pointOutsideImage@{pointOutsideImage}!PCodePointer@{PCodePointer}}
\subsubsection[{pointOutsideImage}]{\setlength{\rightskip}{0pt plus 5cm}bool PCodePointer::pointOutsideImage (PPoint {\em point})}}
\label{classPCodePointer_a9e836354b61f96ed39208c8b37d57d3}


Determinuje, czy punkt o zadanych współrzędnych mieści się w obrazie kodu. \begin{Desc}
\item[Parametry:]
\begin{description}
\item[{\em point}]wskazany punkt \end{description}
\end{Desc}
\begin{Desc}
\item[Zwraca:]czy mieści się w obrazie kodu \end{Desc}
\hypertarget{classPCodePointer_b5c13bf294dddd672ca9fe7675eb258b}{
\index{PCodePointer@{PCodePointer}!setCodelChooser@{setCodelChooser}}
\index{setCodelChooser@{setCodelChooser}!PCodePointer@{PCodePointer}}
\subsubsection[{setCodelChooser}]{\setlength{\rightskip}{0pt plus 5cm}void PCodePointer::setCodelChooser ({\bf PCodelChooserValues} {\em cc})\hspace{0.3cm}{\tt  \mbox{[}protected\mbox{]}}}}
\label{classPCodePointer_b5c13bf294dddd672ca9fe7675eb258b}


Ustala wartość CC. \begin{Desc}
\item[Parametry:]
\begin{description}
\item[{\em cc}]wartość CC \end{description}
\end{Desc}
\hypertarget{classPCodePointer_117bf9322b4fdcb189c41a6f2f113a4b}{
\index{PCodePointer@{PCodePointer}!setCoordinates@{setCoordinates}}
\index{setCoordinates@{setCoordinates}!PCodePointer@{PCodePointer}}
\subsubsection[{setCoordinates}]{\setlength{\rightskip}{0pt plus 5cm}void PCodePointer::setCoordinates (PPoint {\em new\_\-point})}}
\label{classPCodePointer_117bf9322b4fdcb189c41a6f2f113a4b}


Ustala współrzędne głowicy. \begin{Desc}
\item[Parametry:]
\begin{description}
\item[{\em new\_\-point}]punkt (nowe współrzędne) \end{description}
\end{Desc}
\hypertarget{classPCodePointer_9bcf58c97f704e3be2deefb588f91312}{
\index{PCodePointer@{PCodePointer}!setCoordinateX@{setCoordinateX}}
\index{setCoordinateX@{setCoordinateX}!PCodePointer@{PCodePointer}}
\subsubsection[{setCoordinateX}]{\setlength{\rightskip}{0pt plus 5cm}void PCodePointer::setCoordinateX (int {\em newX})}}
\label{classPCodePointer_9bcf58c97f704e3be2deefb588f91312}


Ustala współrzędną X głowicy. \begin{Desc}
\item[Parametry:]
\begin{description}
\item[{\em newX}]nowa wartość współrzędnej X \end{description}
\end{Desc}
\hypertarget{classPCodePointer_9fd77f14e39cc30f8cfc583ce97aa2f2}{
\index{PCodePointer@{PCodePointer}!setCoordinateY@{setCoordinateY}}
\index{setCoordinateY@{setCoordinateY}!PCodePointer@{PCodePointer}}
\subsubsection[{setCoordinateY}]{\setlength{\rightskip}{0pt plus 5cm}void PCodePointer::setCoordinateY (int {\em newY})}}
\label{classPCodePointer_9fd77f14e39cc30f8cfc583ce97aa2f2}


Ustala współrzędną Y głowicy. \begin{Desc}
\item[Parametry:]
\begin{description}
\item[{\em newY}]nowa wartość współrzędnej Y \end{description}
\end{Desc}
\hypertarget{classPCodePointer_0b97bd6b4383e976e54c5738178c7815}{
\index{PCodePointer@{PCodePointer}!setDirectionPointer@{setDirectionPointer}}
\index{setDirectionPointer@{setDirectionPointer}!PCodePointer@{PCodePointer}}
\subsubsection[{setDirectionPointer}]{\setlength{\rightskip}{0pt plus 5cm}void PCodePointer::setDirectionPointer ({\bf PDirectionPointerValues} {\em dp})\hspace{0.3cm}{\tt  \mbox{[}protected\mbox{]}}}}
\label{classPCodePointer_0b97bd6b4383e976e54c5738178c7815}


Ustala wartość DP. \begin{Desc}
\item[Parametry:]
\begin{description}
\item[{\em dp}]wartość DP \end{description}
\end{Desc}
\hypertarget{classPCodePointer_d738019c8cb766c0863e2c00622d9fc9}{
\index{PCodePointer@{PCodePointer}!setVerbosity@{setVerbosity}}
\index{setVerbosity@{setVerbosity}!PCodePointer@{PCodePointer}}
\subsubsection[{setVerbosity}]{\setlength{\rightskip}{0pt plus 5cm}void PCodePointer::setVerbosity (bool {\em verbosity})}}
\label{classPCodePointer_d738019c8cb766c0863e2c00622d9fc9}


Ustala tryb gadatliwy. \begin{Desc}
\item[Parametry:]
\begin{description}
\item[{\em verbosity}]tryb gadatliwy \end{description}
\end{Desc}
\hypertarget{classPCodePointer_5b34ab0f6bb3ddb3bd7e44a7d9a613ca}{
\index{PCodePointer@{PCodePointer}!toggle@{toggle}}
\index{toggle@{toggle}!PCodePointer@{PCodePointer}}
\subsubsection[{toggle}]{\setlength{\rightskip}{0pt plus 5cm}void PCodePointer::toggle ()}}
\label{classPCodePointer_5b34ab0f6bb3ddb3bd7e44a7d9a613ca}


Zmienia wartość CC oraz DP w sytuacji, gdy głowica nie mogła iść dalej w dotychczasowym kierunku. \hypertarget{classPCodePointer_56d2ef632779fbe64937030e82b027d1}{
\index{PCodePointer@{PCodePointer}!toggleCodelChooser@{toggleCodelChooser}}
\index{toggleCodelChooser@{toggleCodelChooser}!PCodePointer@{PCodePointer}}
\subsubsection[{toggleCodelChooser}]{\setlength{\rightskip}{0pt plus 5cm}void PCodePointer::toggleCodelChooser ()}}
\label{classPCodePointer_56d2ef632779fbe64937030e82b027d1}


Zmienia wartość CC w sytuacji, gdy głowica nie mogła iść dalej w dotychczasowym kierunku. Używane wewnątrz metody \hyperlink{classPCodePointer_5b34ab0f6bb3ddb3bd7e44a7d9a613ca}{toggle()}. \hypertarget{classPCodePointer_4aab1f30e01bb0fb3d78b5e6aa93535c}{
\index{PCodePointer@{PCodePointer}!toggleDirectionPointer@{toggleDirectionPointer}}
\index{toggleDirectionPointer@{toggleDirectionPointer}!PCodePointer@{PCodePointer}}
\subsubsection[{toggleDirectionPointer}]{\setlength{\rightskip}{0pt plus 5cm}void PCodePointer::toggleDirectionPointer ()}}
\label{classPCodePointer_4aab1f30e01bb0fb3d78b5e6aa93535c}


Zmienia wartość DP w sytuacji, gdy głowica nie mogła iść dalej w dotychczasowym kierunku. Używane wewnątrz metody \hyperlink{classPCodePointer_5b34ab0f6bb3ddb3bd7e44a7d9a613ca}{toggle()}. \hypertarget{classPCodePointer_3f43e8205a0554bf7baa3821e01ab4a2}{
\index{PCodePointer@{PCodePointer}!turnDirectionPointerAnticlockwise@{turnDirectionPointerAnticlockwise}}
\index{turnDirectionPointerAnticlockwise@{turnDirectionPointerAnticlockwise}!PCodePointer@{PCodePointer}}
\subsubsection[{turnDirectionPointerAnticlockwise}]{\setlength{\rightskip}{0pt plus 5cm}void PCodePointer::turnDirectionPointerAnticlockwise ()\hspace{0.3cm}{\tt  \mbox{[}protected\mbox{]}}}}
\label{classPCodePointer_3f43e8205a0554bf7baa3821e01ab4a2}


Obraca DP przeciwnie do ruchu wskazówek zegara raz. \hypertarget{classPCodePointer_a3ad29e6327d54faf0b081892e720aca}{
\index{PCodePointer@{PCodePointer}!turnDirectionPointerClockwise@{turnDirectionPointerClockwise}}
\index{turnDirectionPointerClockwise@{turnDirectionPointerClockwise}!PCodePointer@{PCodePointer}}
\subsubsection[{turnDirectionPointerClockwise}]{\setlength{\rightskip}{0pt plus 5cm}void PCodePointer::turnDirectionPointerClockwise ()\hspace{0.3cm}{\tt  \mbox{[}protected\mbox{]}}}}
\label{classPCodePointer_a3ad29e6327d54faf0b081892e720aca}


Obraca DP zgodnie z ruchem wskazówek zegara raz. 

\subsection{Dokumentacja atrybutów składowych}
\hypertarget{classPCodePointer_a3d7ea2563c9e1363ddcb0fb19b35f5a}{
\index{PCodePointer@{PCodePointer}!codel\_\-chooser@{codel\_\-chooser}}
\index{codel\_\-chooser@{codel\_\-chooser}!PCodePointer@{PCodePointer}}
\subsubsection[{codel\_\-chooser}]{\setlength{\rightskip}{0pt plus 5cm}{\bf PCodelChooserValues} {\bf PCodePointer::codel\_\-chooser}\hspace{0.3cm}{\tt  \mbox{[}protected\mbox{]}}}}
\label{classPCodePointer_a3d7ea2563c9e1363ddcb0fb19b35f5a}


Codel chooser (CC, selektor kodeli). \hypertarget{classPCodePointer_b285be0011b9b5c8b35bd1ba27ee935f}{
\index{PCodePointer@{PCodePointer}!direction\_\-pointer@{direction\_\-pointer}}
\index{direction\_\-pointer@{direction\_\-pointer}!PCodePointer@{PCodePointer}}
\subsubsection[{direction\_\-pointer}]{\setlength{\rightskip}{0pt plus 5cm}{\bf PDirectionPointerValues} {\bf PCodePointer::direction\_\-pointer}\hspace{0.3cm}{\tt  \mbox{[}protected\mbox{]}}}}
\label{classPCodePointer_b285be0011b9b5c8b35bd1ba27ee935f}


Direction pointer (DP, wskaźnik kierunku). \hypertarget{classPCodePointer_9f6689e6425046abd85001f915ba1221}{
\index{PCodePointer@{PCodePointer}!image@{image}}
\index{image@{image}!PCodePointer@{PCodePointer}}
\subsubsection[{image}]{\setlength{\rightskip}{0pt plus 5cm}QImage$\ast$ {\bf PCodePointer::image}\hspace{0.3cm}{\tt  \mbox{[}protected\mbox{]}}}}
\label{classPCodePointer_9f6689e6425046abd85001f915ba1221}


Referencja do obrazu kodu, po którym porusza się głowica. 

Dokumentacja dla tej klasy została wygenerowana z plików:\begin{CompactItemize}
\item 
src/core/\hyperlink{pcodepointer_8h}{pcodepointer.h}\item 
src/core/\hyperlink{pcodepointer_8cpp}{pcodepointer.cpp}\end{CompactItemize}

\hypertarget{classPColorManager}{
\section{Dokumentacja klasy PColorManager}
\label{classPColorManager}\index{PColorManager@{PColorManager}}
}
Maszyna kodu.  


{\tt \#include $<$pcolormanager.h$>$}

\subsection*{Metody publiczne}
\begin{CompactItemize}
\item 
void \hyperlink{classPColorManager_ee267bb2056457f73dcf1115e7d3c973}{setVerbosity} (bool)
\item 
int \hyperlink{classPColorManager_36a84d6e85c87edf65fdb511690ed434}{getInstructionIndex} (QRgb, QRgb)
\item 
\hyperlink{penums_8h_4fb01e50a2da85245f7cea7856eca8ec}{PStdColors} \hyperlink{classPColorManager_27aeb6623eaf527db2d7428ab038958c}{getColorName} (QRgb)
\item 
void \hyperlink{classPColorManager_e9793e8da1800653b11609376098695c}{initColorValues} ()
\item 
void \hyperlink{classPColorManager_83e07cad55aed7d9133e42a71d514d26}{\_\-\_\-dev\_\-\_\-printColor} (\hyperlink{penums_8h_4fb01e50a2da85245f7cea7856eca8ec}{PStdColors})
\end{CompactItemize}
\subsection*{Atrybuty publiczne}
\begin{CompactItemize}
\item 
QRgb \hyperlink{classPColorManager_cbc4c8b260f0ba856d8454ae2f9320d5}{LIGHT\_\-RED}
\item 
QRgb \hyperlink{classPColorManager_c6f06fc1bb188fe7d7f573ff3c46df8d}{NORMAL\_\-RED}
\item 
QRgb \hyperlink{classPColorManager_3690840932799e3d910dd3d673f90363}{DARK\_\-RED}
\item 
QRgb \hyperlink{classPColorManager_945305800d979369bb29d9f3ec5fe7f3}{LIGHT\_\-YELLOW}
\item 
QRgb \hyperlink{classPColorManager_a1086408be008b97df57f579702d624e}{NORMAL\_\-YELLOW}
\item 
QRgb \hyperlink{classPColorManager_347da61b74a87e7eed4b8849a8128e11}{DARK\_\-YELLOW}
\item 
QRgb \hyperlink{classPColorManager_45eba1a3facb6995710a2d1e9026ca7c}{LIGHT\_\-GREEN}
\item 
QRgb \hyperlink{classPColorManager_a2e75d69d65d6a7f87a470c0782a7624}{NORMAL\_\-GREEN}
\item 
QRgb \hyperlink{classPColorManager_ec8d0fe14a68b42be5c5d8d77f914411}{DARK\_\-GREEN}
\item 
QRgb \hyperlink{classPColorManager_4b0005532d50c98afdbde28fb3081e47}{LIGHT\_\-CYAN}
\item 
QRgb \hyperlink{classPColorManager_9b51000ba56bf9a5f0581fd6c9a51ab6}{NORMAL\_\-CYAN}
\item 
QRgb \hyperlink{classPColorManager_d7dcde24002d9fa0ee9600d37720a930}{DARK\_\-CYAN}
\item 
QRgb \hyperlink{classPColorManager_861249c4294f958c0043a1d1b7468435}{LIGHT\_\-BLUE}
\item 
QRgb \hyperlink{classPColorManager_530893a4eed2d71ce67dff0d500bd0d6}{NORMAL\_\-BLUE}
\item 
QRgb \hyperlink{classPColorManager_2d7a0c5c065fa5b662cb53748e5347ed}{DARK\_\-BLUE}
\item 
QRgb \hyperlink{classPColorManager_435503e14622c3bf22d82a4c5ac12884}{LIGHT\_\-MAGENTA}
\item 
QRgb \hyperlink{classPColorManager_28bb798b6307bd6b6bdd59204f3dc131}{NORMAL\_\-MAGENTA}
\item 
QRgb \hyperlink{classPColorManager_eabafba23c771c3ad19f67938e11f68b}{DARK\_\-MAGENTA}
\item 
QRgb \hyperlink{classPColorManager_c15cc3f24bec369e9e30fee77ca8899a}{WHITE}
\item 
QRgb \hyperlink{classPColorManager_13278cb96706ed743a3f86a6cf4232cb}{BLACK}
\end{CompactItemize}


\subsection{Opis szczegółowy}
Maszyna kodu. 

Klasa odpowiedzialna za właściwe interpretowanie kolorów, determinowanie czy kolor jest standardowy, jeśli niestandardowy - przydziela go do odpowiedniego koloru standardowego wg odpowiedniego algorytmu. 

\subsection{Dokumentacja funkcji składowych}
\hypertarget{classPColorManager_83e07cad55aed7d9133e42a71d514d26}{
\index{PColorManager@{PColorManager}!\_\-\_\-dev\_\-\_\-printColor@{\_\-\_\-dev\_\-\_\-printColor}}
\index{\_\-\_\-dev\_\-\_\-printColor@{\_\-\_\-dev\_\-\_\-printColor}!PColorManager@{PColorManager}}
\subsubsection[{\_\-\_\-dev\_\-\_\-printColor}]{\setlength{\rightskip}{0pt plus 5cm}void PColorManager::\_\-\_\-dev\_\-\_\-printColor ({\bf PStdColors} {\em color})}}
\label{classPColorManager_83e07cad55aed7d9133e42a71d514d26}


METODA TESTOWA. Wyświetla nazwę koloru \begin{Desc}
\item[Parametry:]
\begin{description}
\item[{\em color}]kolor \end{description}
\end{Desc}
\hypertarget{classPColorManager_27aeb6623eaf527db2d7428ab038958c}{
\index{PColorManager@{PColorManager}!getColorName@{getColorName}}
\index{getColorName@{getColorName}!PColorManager@{PColorManager}}
\subsubsection[{getColorName}]{\setlength{\rightskip}{0pt plus 5cm}{\bf PStdColors} PColorManager::getColorName (QRgb {\em color})}}
\label{classPColorManager_27aeb6623eaf527db2d7428ab038958c}


Zwraca kolor jako element enumeracji dla zadanej wartości koloru typu QRgb. \begin{Desc}
\item[Parametry:]
\begin{description}
\item[{\em color}]zadany kolor (wartość) \end{description}
\end{Desc}
\begin{Desc}
\item[Zwraca:]kolor (element enumeracji) \end{Desc}
\hypertarget{classPColorManager_36a84d6e85c87edf65fdb511690ed434}{
\index{PColorManager@{PColorManager}!getInstructionIndex@{getInstructionIndex}}
\index{getInstructionIndex@{getInstructionIndex}!PColorManager@{PColorManager}}
\subsubsection[{getInstructionIndex}]{\setlength{\rightskip}{0pt plus 5cm}int PColorManager::getInstructionIndex (QRgb {\em old\_\-color}, \/  QRgb {\em new\_\-color})}}
\label{classPColorManager_36a84d6e85c87edf65fdb511690ed434}


Wyznacza instrukcję Pieat jaka ma zostać wykonana na podstawie dwóch kolorów bloków: pierwszy - który głowica opuściła i drugi - do którego głowica weszła. Metoda implementuje algorytm wyznaczania instrukcji opisany w tabeli 1.2 w rozdziale 1 pracy magisterskiej. \begin{Desc}
\item[Parametry:]
\begin{description}
\item[{\em old\_\-color}]kolor bloku który głowica opuściła \item[{\em new\_\-color}]kolor bloku do którego głowica weszła \end{description}
\end{Desc}
\begin{Desc}
\item[Zwraca:]indeks instrukcji która powinna zostać wykonana \end{Desc}
\hypertarget{classPColorManager_e9793e8da1800653b11609376098695c}{
\index{PColorManager@{PColorManager}!initColorValues@{initColorValues}}
\index{initColorValues@{initColorValues}!PColorManager@{PColorManager}}
\subsubsection[{initColorValues}]{\setlength{\rightskip}{0pt plus 5cm}void PColorManager::initColorValues ()}}
\label{classPColorManager_e9793e8da1800653b11609376098695c}


Inicjuje wartości zmiennych typu QRgb reprezentujących wszystkie 20 kolorów standardowych Pieta. Są one potem wykorzystywane praktycznie cały czas: do wyznaczania instrukcji które powinny być wykonane, do sprawdzania białych lub czarnych bloków. \hypertarget{classPColorManager_ee267bb2056457f73dcf1115e7d3c973}{
\index{PColorManager@{PColorManager}!setVerbosity@{setVerbosity}}
\index{setVerbosity@{setVerbosity}!PColorManager@{PColorManager}}
\subsubsection[{setVerbosity}]{\setlength{\rightskip}{0pt plus 5cm}void PColorManager::setVerbosity (bool {\em verbosity})}}
\label{classPColorManager_ee267bb2056457f73dcf1115e7d3c973}


Ustala tryb gadatliwy. \begin{Desc}
\item[Parametry:]
\begin{description}
\item[{\em verbosity}]tryb gadatliwy \end{description}
\end{Desc}


\subsection{Dokumentacja atrybutów składowych}
\hypertarget{classPColorManager_13278cb96706ed743a3f86a6cf4232cb}{
\index{PColorManager@{PColorManager}!BLACK@{BLACK}}
\index{BLACK@{BLACK}!PColorManager@{PColorManager}}
\subsubsection[{BLACK}]{\setlength{\rightskip}{0pt plus 5cm}QRgb {\bf PColorManager::BLACK}}}
\label{classPColorManager_13278cb96706ed743a3f86a6cf4232cb}


kolor czarny \hypertarget{classPColorManager_2d7a0c5c065fa5b662cb53748e5347ed}{
\index{PColorManager@{PColorManager}!DARK\_\-BLUE@{DARK\_\-BLUE}}
\index{DARK\_\-BLUE@{DARK\_\-BLUE}!PColorManager@{PColorManager}}
\subsubsection[{DARK\_\-BLUE}]{\setlength{\rightskip}{0pt plus 5cm}QRgb {\bf PColorManager::DARK\_\-BLUE}}}
\label{classPColorManager_2d7a0c5c065fa5b662cb53748e5347ed}


kolor ciemny niebieski \hypertarget{classPColorManager_d7dcde24002d9fa0ee9600d37720a930}{
\index{PColorManager@{PColorManager}!DARK\_\-CYAN@{DARK\_\-CYAN}}
\index{DARK\_\-CYAN@{DARK\_\-CYAN}!PColorManager@{PColorManager}}
\subsubsection[{DARK\_\-CYAN}]{\setlength{\rightskip}{0pt plus 5cm}QRgb {\bf PColorManager::DARK\_\-CYAN}}}
\label{classPColorManager_d7dcde24002d9fa0ee9600d37720a930}


kolor ciemny błękitny \hypertarget{classPColorManager_ec8d0fe14a68b42be5c5d8d77f914411}{
\index{PColorManager@{PColorManager}!DARK\_\-GREEN@{DARK\_\-GREEN}}
\index{DARK\_\-GREEN@{DARK\_\-GREEN}!PColorManager@{PColorManager}}
\subsubsection[{DARK\_\-GREEN}]{\setlength{\rightskip}{0pt plus 5cm}QRgb {\bf PColorManager::DARK\_\-GREEN}}}
\label{classPColorManager_ec8d0fe14a68b42be5c5d8d77f914411}


kolor ziemny zielony \hypertarget{classPColorManager_eabafba23c771c3ad19f67938e11f68b}{
\index{PColorManager@{PColorManager}!DARK\_\-MAGENTA@{DARK\_\-MAGENTA}}
\index{DARK\_\-MAGENTA@{DARK\_\-MAGENTA}!PColorManager@{PColorManager}}
\subsubsection[{DARK\_\-MAGENTA}]{\setlength{\rightskip}{0pt plus 5cm}QRgb {\bf PColorManager::DARK\_\-MAGENTA}}}
\label{classPColorManager_eabafba23c771c3ad19f67938e11f68b}


kolor ciemny magenta \hypertarget{classPColorManager_3690840932799e3d910dd3d673f90363}{
\index{PColorManager@{PColorManager}!DARK\_\-RED@{DARK\_\-RED}}
\index{DARK\_\-RED@{DARK\_\-RED}!PColorManager@{PColorManager}}
\subsubsection[{DARK\_\-RED}]{\setlength{\rightskip}{0pt plus 5cm}QRgb {\bf PColorManager::DARK\_\-RED}}}
\label{classPColorManager_3690840932799e3d910dd3d673f90363}


kolor ciemny czerwony \hypertarget{classPColorManager_347da61b74a87e7eed4b8849a8128e11}{
\index{PColorManager@{PColorManager}!DARK\_\-YELLOW@{DARK\_\-YELLOW}}
\index{DARK\_\-YELLOW@{DARK\_\-YELLOW}!PColorManager@{PColorManager}}
\subsubsection[{DARK\_\-YELLOW}]{\setlength{\rightskip}{0pt plus 5cm}QRgb {\bf PColorManager::DARK\_\-YELLOW}}}
\label{classPColorManager_347da61b74a87e7eed4b8849a8128e11}


kolor ciemny żółty \hypertarget{classPColorManager_861249c4294f958c0043a1d1b7468435}{
\index{PColorManager@{PColorManager}!LIGHT\_\-BLUE@{LIGHT\_\-BLUE}}
\index{LIGHT\_\-BLUE@{LIGHT\_\-BLUE}!PColorManager@{PColorManager}}
\subsubsection[{LIGHT\_\-BLUE}]{\setlength{\rightskip}{0pt plus 5cm}QRgb {\bf PColorManager::LIGHT\_\-BLUE}}}
\label{classPColorManager_861249c4294f958c0043a1d1b7468435}


kolor jasny niebieski \hypertarget{classPColorManager_4b0005532d50c98afdbde28fb3081e47}{
\index{PColorManager@{PColorManager}!LIGHT\_\-CYAN@{LIGHT\_\-CYAN}}
\index{LIGHT\_\-CYAN@{LIGHT\_\-CYAN}!PColorManager@{PColorManager}}
\subsubsection[{LIGHT\_\-CYAN}]{\setlength{\rightskip}{0pt plus 5cm}QRgb {\bf PColorManager::LIGHT\_\-CYAN}}}
\label{classPColorManager_4b0005532d50c98afdbde28fb3081e47}


kolor jasny błękitny \hypertarget{classPColorManager_45eba1a3facb6995710a2d1e9026ca7c}{
\index{PColorManager@{PColorManager}!LIGHT\_\-GREEN@{LIGHT\_\-GREEN}}
\index{LIGHT\_\-GREEN@{LIGHT\_\-GREEN}!PColorManager@{PColorManager}}
\subsubsection[{LIGHT\_\-GREEN}]{\setlength{\rightskip}{0pt plus 5cm}QRgb {\bf PColorManager::LIGHT\_\-GREEN}}}
\label{classPColorManager_45eba1a3facb6995710a2d1e9026ca7c}


kolor jasny zielony \hypertarget{classPColorManager_435503e14622c3bf22d82a4c5ac12884}{
\index{PColorManager@{PColorManager}!LIGHT\_\-MAGENTA@{LIGHT\_\-MAGENTA}}
\index{LIGHT\_\-MAGENTA@{LIGHT\_\-MAGENTA}!PColorManager@{PColorManager}}
\subsubsection[{LIGHT\_\-MAGENTA}]{\setlength{\rightskip}{0pt plus 5cm}QRgb {\bf PColorManager::LIGHT\_\-MAGENTA}}}
\label{classPColorManager_435503e14622c3bf22d82a4c5ac12884}


kolor jasny magenta \hypertarget{classPColorManager_cbc4c8b260f0ba856d8454ae2f9320d5}{
\index{PColorManager@{PColorManager}!LIGHT\_\-RED@{LIGHT\_\-RED}}
\index{LIGHT\_\-RED@{LIGHT\_\-RED}!PColorManager@{PColorManager}}
\subsubsection[{LIGHT\_\-RED}]{\setlength{\rightskip}{0pt plus 5cm}QRgb {\bf PColorManager::LIGHT\_\-RED}}}
\label{classPColorManager_cbc4c8b260f0ba856d8454ae2f9320d5}


kolor jasny czerwony \hypertarget{classPColorManager_945305800d979369bb29d9f3ec5fe7f3}{
\index{PColorManager@{PColorManager}!LIGHT\_\-YELLOW@{LIGHT\_\-YELLOW}}
\index{LIGHT\_\-YELLOW@{LIGHT\_\-YELLOW}!PColorManager@{PColorManager}}
\subsubsection[{LIGHT\_\-YELLOW}]{\setlength{\rightskip}{0pt plus 5cm}QRgb {\bf PColorManager::LIGHT\_\-YELLOW}}}
\label{classPColorManager_945305800d979369bb29d9f3ec5fe7f3}


kolor jasny żółty \hypertarget{classPColorManager_530893a4eed2d71ce67dff0d500bd0d6}{
\index{PColorManager@{PColorManager}!NORMAL\_\-BLUE@{NORMAL\_\-BLUE}}
\index{NORMAL\_\-BLUE@{NORMAL\_\-BLUE}!PColorManager@{PColorManager}}
\subsubsection[{NORMAL\_\-BLUE}]{\setlength{\rightskip}{0pt plus 5cm}QRgb {\bf PColorManager::NORMAL\_\-BLUE}}}
\label{classPColorManager_530893a4eed2d71ce67dff0d500bd0d6}


kolor niebieski \hypertarget{classPColorManager_9b51000ba56bf9a5f0581fd6c9a51ab6}{
\index{PColorManager@{PColorManager}!NORMAL\_\-CYAN@{NORMAL\_\-CYAN}}
\index{NORMAL\_\-CYAN@{NORMAL\_\-CYAN}!PColorManager@{PColorManager}}
\subsubsection[{NORMAL\_\-CYAN}]{\setlength{\rightskip}{0pt plus 5cm}QRgb {\bf PColorManager::NORMAL\_\-CYAN}}}
\label{classPColorManager_9b51000ba56bf9a5f0581fd6c9a51ab6}


kolor błękitny \hypertarget{classPColorManager_a2e75d69d65d6a7f87a470c0782a7624}{
\index{PColorManager@{PColorManager}!NORMAL\_\-GREEN@{NORMAL\_\-GREEN}}
\index{NORMAL\_\-GREEN@{NORMAL\_\-GREEN}!PColorManager@{PColorManager}}
\subsubsection[{NORMAL\_\-GREEN}]{\setlength{\rightskip}{0pt plus 5cm}QRgb {\bf PColorManager::NORMAL\_\-GREEN}}}
\label{classPColorManager_a2e75d69d65d6a7f87a470c0782a7624}


kolor zielony \hypertarget{classPColorManager_28bb798b6307bd6b6bdd59204f3dc131}{
\index{PColorManager@{PColorManager}!NORMAL\_\-MAGENTA@{NORMAL\_\-MAGENTA}}
\index{NORMAL\_\-MAGENTA@{NORMAL\_\-MAGENTA}!PColorManager@{PColorManager}}
\subsubsection[{NORMAL\_\-MAGENTA}]{\setlength{\rightskip}{0pt plus 5cm}QRgb {\bf PColorManager::NORMAL\_\-MAGENTA}}}
\label{classPColorManager_28bb798b6307bd6b6bdd59204f3dc131}


kolor magenta \hypertarget{classPColorManager_c6f06fc1bb188fe7d7f573ff3c46df8d}{
\index{PColorManager@{PColorManager}!NORMAL\_\-RED@{NORMAL\_\-RED}}
\index{NORMAL\_\-RED@{NORMAL\_\-RED}!PColorManager@{PColorManager}}
\subsubsection[{NORMAL\_\-RED}]{\setlength{\rightskip}{0pt plus 5cm}QRgb {\bf PColorManager::NORMAL\_\-RED}}}
\label{classPColorManager_c6f06fc1bb188fe7d7f573ff3c46df8d}


kolor czerwony \hypertarget{classPColorManager_a1086408be008b97df57f579702d624e}{
\index{PColorManager@{PColorManager}!NORMAL\_\-YELLOW@{NORMAL\_\-YELLOW}}
\index{NORMAL\_\-YELLOW@{NORMAL\_\-YELLOW}!PColorManager@{PColorManager}}
\subsubsection[{NORMAL\_\-YELLOW}]{\setlength{\rightskip}{0pt plus 5cm}QRgb {\bf PColorManager::NORMAL\_\-YELLOW}}}
\label{classPColorManager_a1086408be008b97df57f579702d624e}


kolor żółty \hypertarget{classPColorManager_c15cc3f24bec369e9e30fee77ca8899a}{
\index{PColorManager@{PColorManager}!WHITE@{WHITE}}
\index{WHITE@{WHITE}!PColorManager@{PColorManager}}
\subsubsection[{WHITE}]{\setlength{\rightskip}{0pt plus 5cm}QRgb {\bf PColorManager::WHITE}}}
\label{classPColorManager_c15cc3f24bec369e9e30fee77ca8899a}


kolor biały 

Dokumentacja dla tej klasy została wygenerowana z plików:\begin{CompactItemize}
\item 
src/core/\hyperlink{pcolormanager_8h}{pcolormanager.h}\item 
src/core/\hyperlink{pcolormanager_8cpp}{pcolormanager.cpp}\end{CompactItemize}

\hypertarget{classPConsole}{
\section{Dokumentacja klasy PConsole}
\label{classPConsole}\index{PConsole@{PConsole}}
}
Konsola wejścia/wyjścia.  


{\tt \#include $<$pconsole.h$>$}

\subsection*{Metody publiczne}
\begin{CompactItemize}
\item 
\hyperlink{classPConsole_c783e5f2b3c5a0bbde30171ab1226be4}{PConsole} ()
\item 
\hyperlink{classPConsole_c4a28c9ee5bfd46977b4339672212aca}{$\sim$PConsole} ()
\item 
void \hyperlink{classPConsole_b5446c4f888fe9552788a2e757f23b6b}{setVerbosity} (bool)
\item 
void \hyperlink{classPConsole_1c0c7c5be680cdb2e456e3e34798c7a6}{printChar} (int)
\item 
void \hyperlink{classPConsole_744226ceb9724625cadf1124a2b87f55}{printNumber} (int)
\item 
int \hyperlink{classPConsole_5bf368ed172de8c7e8221973b3c613c3}{readChar} ()
\item 
int \hyperlink{classPConsole_fbe2b8e906db6d9fc70397d8a9b4105c}{readNumber} ()
\end{CompactItemize}


\subsection{Opis szczegółowy}
Konsola wejścia/wyjścia. 

Klasa realizująca wszystkie operacje wejścia wyjścia związane z instrukcjami Pieta. 

\subsection{Dokumentacja konstruktora i destruktora}
\hypertarget{classPConsole_c783e5f2b3c5a0bbde30171ab1226be4}{
\index{PConsole@{PConsole}!PConsole@{PConsole}}
\index{PConsole@{PConsole}!PConsole@{PConsole}}
\subsubsection[{PConsole}]{\setlength{\rightskip}{0pt plus 5cm}PConsole::PConsole ()}}
\label{classPConsole_c783e5f2b3c5a0bbde30171ab1226be4}


Konstruktor konsoli wejścia/wyjścia. Nie robi nic szczególnego. \hypertarget{classPConsole_c4a28c9ee5bfd46977b4339672212aca}{
\index{PConsole@{PConsole}!$\sim$PConsole@{$\sim$PConsole}}
\index{$\sim$PConsole@{$\sim$PConsole}!PConsole@{PConsole}}
\subsubsection[{$\sim$PConsole}]{\setlength{\rightskip}{0pt plus 5cm}PConsole::$\sim$PConsole ()}}
\label{classPConsole_c4a28c9ee5bfd46977b4339672212aca}


Destruktor konsoli wejścia/wyjścia. Nie robi nic szczególnego. 

\subsection{Dokumentacja funkcji składowych}
\hypertarget{classPConsole_1c0c7c5be680cdb2e456e3e34798c7a6}{
\index{PConsole@{PConsole}!printChar@{printChar}}
\index{printChar@{printChar}!PConsole@{PConsole}}
\subsubsection[{printChar}]{\setlength{\rightskip}{0pt plus 5cm}void PConsole::printChar (int {\em I})}}
\label{classPConsole_1c0c7c5be680cdb2e456e3e34798c7a6}


Wyświetla znak na konsolę. \begin{Desc}
\item[Parametry:]
\begin{description}
\item[{\em I}]znak reprezentowany maszynowo przez liczbę \end{description}
\end{Desc}
\hypertarget{classPConsole_744226ceb9724625cadf1124a2b87f55}{
\index{PConsole@{PConsole}!printNumber@{printNumber}}
\index{printNumber@{printNumber}!PConsole@{PConsole}}
\subsubsection[{printNumber}]{\setlength{\rightskip}{0pt plus 5cm}void PConsole::printNumber (int {\em I})}}
\label{classPConsole_744226ceb9724625cadf1124a2b87f55}


Wyświetla liczbę na konsolę. \begin{Desc}
\item[Parametry:]
\begin{description}
\item[{\em I}]liczba \end{description}
\end{Desc}
\hypertarget{classPConsole_5bf368ed172de8c7e8221973b3c613c3}{
\index{PConsole@{PConsole}!readChar@{readChar}}
\index{readChar@{readChar}!PConsole@{PConsole}}
\subsubsection[{readChar}]{\setlength{\rightskip}{0pt plus 5cm}int PConsole::readChar ()}}
\label{classPConsole_5bf368ed172de8c7e8221973b3c613c3}


Wczytuje z koncoli znak. \begin{Desc}
\item[Zwraca:]wczytany znak reprezentowany maszynowo przez liczbę \end{Desc}
\hypertarget{classPConsole_fbe2b8e906db6d9fc70397d8a9b4105c}{
\index{PConsole@{PConsole}!readNumber@{readNumber}}
\index{readNumber@{readNumber}!PConsole@{PConsole}}
\subsubsection[{readNumber}]{\setlength{\rightskip}{0pt plus 5cm}int PConsole::readNumber ()}}
\label{classPConsole_fbe2b8e906db6d9fc70397d8a9b4105c}


Wczytuje z koncoli liczbę. \begin{Desc}
\item[Zwraca:]wczytana liczba \end{Desc}
\hypertarget{classPConsole_b5446c4f888fe9552788a2e757f23b6b}{
\index{PConsole@{PConsole}!setVerbosity@{setVerbosity}}
\index{setVerbosity@{setVerbosity}!PConsole@{PConsole}}
\subsubsection[{setVerbosity}]{\setlength{\rightskip}{0pt plus 5cm}void PConsole::setVerbosity (bool {\em verbosity})}}
\label{classPConsole_b5446c4f888fe9552788a2e757f23b6b}


Ustala tryb gadatliwy. \begin{Desc}
\item[Parametry:]
\begin{description}
\item[{\em verbosity}]tryb gadatliwy \end{description}
\end{Desc}


Dokumentacja dla tej klasy została wygenerowana z plików:\begin{CompactItemize}
\item 
src/core/\hyperlink{pconsole_8h}{pconsole.h}\item 
src/core/\hyperlink{pconsole_8cpp}{pconsole.cpp}\end{CompactItemize}

\hypertarget{classPVirtualMachine}{
\section{Dokumentacja klasy PVirtualMachine}
\label{classPVirtualMachine}\index{PVirtualMachine@{PVirtualMachine}}
}
Wirtualna maszyna Pieta.  


{\tt \#include $<$pvirtualmachine.h$>$}

\subsection*{Metody publiczne}
\begin{CompactItemize}
\item 
\hyperlink{classPVirtualMachine_0af0326622a3f47b6c2020413cdd3e05}{PVirtualMachine} (QString)
\item 
\hyperlink{classPVirtualMachine_4b24d240f1a45515cf4c8d4df932da62}{$\sim$PVirtualMachine} ()
\item 
bool \hyperlink{classPVirtualMachine_5132bedd5e8480185b48ea6c4783c770}{isReady} ()
\item 
bool \hyperlink{classPVirtualMachine_823195d3ae1615c791dfeb6c8d365a0e}{isRunning} ()
\item 
bool \hyperlink{classPVirtualMachine_6b19e3e51a584b13e79c44fc5b842c2b}{isFinished} ()
\item 
bool \hyperlink{classPVirtualMachine_4b6bb128d63569e855684128b5e09f1c}{startMachine} ()
\item 
bool \hyperlink{classPVirtualMachine_f4cdf5d80b7a169e2dfdebffe2588e4b}{restartMachine} ()
\item 
bool \hyperlink{classPVirtualMachine_fa30f1871832c1af18519db03c397ac0}{stopMachine} ()
\item 
bool \hyperlink{classPVirtualMachine_219cd1d08de6d3d5a9117eecc5348d9c}{isVerbose} ()
\item 
void \hyperlink{classPVirtualMachine_1e48dcf68b3b80d7f6d3c67b3f796e51}{toggleVerbosity} ()
\item 
void \hyperlink{classPVirtualMachine_40e1780e122ce7fe4ebadc526c47fbea}{executeAllInstr} ()
\item 
bool \hyperlink{classPVirtualMachine_7bbb22aabe782de2208139bc01f65050}{executeSingleInstr} ()
\item 
bool \hyperlink{classPVirtualMachine_6daa9fea05af4bfc30b881c893fbd725}{pointIsBlackOrOutside} (\hyperlink{structstruct__point}{PPoint})
\item 
bool \hyperlink{classPVirtualMachine_cae537f70c35487220818e8d18978e07}{pointIsWhite} (\hyperlink{structstruct__point}{PPoint})
\item 
void \hyperlink{classPVirtualMachine_e37d4dfa512c3dd97b9958f013ffb7b3}{slidePointerAcrossWhiteBlock} ()
\item 
void \hyperlink{classPVirtualMachine_4b2ebb998393324c5240dc48fdb14f3a}{slideAcrossWhiteBlock} (\hyperlink{structstruct__point}{PPoint} \&)
\item 
\hyperlink{penums_8h_797d2a71195500b970298fbf455ff83a}{PInstructions} \hyperlink{classPVirtualMachine_29279548406a588bce03e320354cd857}{movePointerAndGetInstructionToExecute} ()
\item 
\hyperlink{penums_8h_797d2a71195500b970298fbf455ff83a}{PInstructions} \hyperlink{classPVirtualMachine_69d12176d3f9719b2bdf93a9f55cefde}{getInstructionByIndex} (int)
\item 
void \hyperlink{classPVirtualMachine_91c888349242ed7700d5ddff5b187c19}{\_\-\_\-dev\_\-\_\-printInstruction} (\hyperlink{penums_8h_797d2a71195500b970298fbf455ff83a}{PInstructions})
\item 
void \hyperlink{classPVirtualMachine_d0ac4994cbf5c218a40442df49a3cd29}{\_\-\_\-dev\_\-\_\-printImageInfo} ()
\item 
void \hyperlink{classPVirtualMachine_9fbb7c0313e22051eebfd2e316ec609b}{\_\-\_\-dev\_\-\_\-printConsole} ()
\end{CompactItemize}
\subsection*{Metody chronione}
\begin{CompactItemize}
\item 
\hyperlink{penums_8h_3b12ec7c990da3435462047165bc0c6d}{PMachineStates} \hyperlink{classPVirtualMachine_699839f4df2106065458127c36b4bfca}{getState} ()
\begin{CompactList}\small\item\em zwraca stan wirtualnej maszyny Salvadora \item\end{CompactList}\item 
void \hyperlink{classPVirtualMachine_bede8c1e068c19b9bc61a957de70cda8}{setState} (\hyperlink{penums_8h_3b12ec7c990da3435462047165bc0c6d}{PMachineStates})
\begin{CompactList}\small\item\em ustawia stan wirtualnej maszyny Salvadora \item\end{CompactList}\item 
void \hyperlink{classPVirtualMachine_513af8673f1430cd04fd0a0e46abedd4}{prepareToExecute} ()
\begin{CompactList}\small\item\em Przygotowanie do uruchomienia/zresetowania wirtualnej maszyny. \item\end{CompactList}\end{CompactItemize}
\subsection*{Atrybuty chronione}
\begin{CompactItemize}
\item 
\hyperlink{classPCalcStack}{PCalcStack} $\ast$ \hyperlink{classPVirtualMachine_fe929445de96589087a96a8f1188fac0}{stack}
\item 
\hyperlink{classPConsole}{PConsole} $\ast$ \hyperlink{classPVirtualMachine_67feeb2d88b4c8b766d10d6f7d8eb65b}{console}
\end{CompactItemize}


\subsection{Opis szczegółowy}
Wirtualna maszyna Pieta. 

Klasa realizująca tzw. \char`\"{}wirtualną maszynę Pieta\char`\"{}, która posiada wszystkie mechanizmy potrzebne do egzekucji kodu Pieta.

Klasa sama w sobie posiada niewiele mechanizmów (precyzyjniej: najmniej jak to było możliwe), aby wszystkie szczegółowe operacje wydzielić dla pozostałych klas w obrębie projektu. Wirtualna maszyna zajmuje się (oprócz tworzenia pomocniczych obiektów) tylko egzekucją pojedynczych instrukcji Pieta (oraz związane z tym zatrzymywanie, uruchamianie, przygotowywanie maszyny).

Jej dwoma najważniejszymi elementami są: stos, na którym przechowywane są tymczasowe wartości, oraz tzw. \char`\"{}maszyna kodu\char`\"{}, odpowiedzialna za pełną interpretację kodu. 

\subsection{Dokumentacja konstruktora i destruktora}
\hypertarget{classPVirtualMachine_0af0326622a3f47b6c2020413cdd3e05}{
\index{PVirtualMachine@{PVirtualMachine}!PVirtualMachine@{PVirtualMachine}}
\index{PVirtualMachine@{PVirtualMachine}!PVirtualMachine@{PVirtualMachine}}
\subsubsection[{PVirtualMachine}]{\setlength{\rightskip}{0pt plus 5cm}PVirtualMachine::PVirtualMachine (QString {\em filename})}}
\label{classPVirtualMachine_0af0326622a3f47b6c2020413cdd3e05}


Konstruktor maszyny wirtualnej interpretującej dowolny program w języku Piet. Tworzy wszystkie pomocnicze obiekty których działanie jest wykorzystywane i koordynowane przez wirtualną maszynę. \hypertarget{classPVirtualMachine_4b24d240f1a45515cf4c8d4df932da62}{
\index{PVirtualMachine@{PVirtualMachine}!$\sim$PVirtualMachine@{$\sim$PVirtualMachine}}
\index{$\sim$PVirtualMachine@{$\sim$PVirtualMachine}!PVirtualMachine@{PVirtualMachine}}
\subsubsection[{$\sim$PVirtualMachine}]{\setlength{\rightskip}{0pt plus 5cm}PVirtualMachine::$\sim$PVirtualMachine ()}}
\label{classPVirtualMachine_4b24d240f1a45515cf4c8d4df932da62}


Destruktor wirtualnej maszyny Pieta. Wywołuje destruktory dla pomocniczych obiektów tworzonych w konstruktorze wirtualnej maszyny. 

\subsection{Dokumentacja funkcji składowych}
\hypertarget{classPVirtualMachine_9fbb7c0313e22051eebfd2e316ec609b}{
\index{PVirtualMachine@{PVirtualMachine}!\_\-\_\-dev\_\-\_\-printConsole@{\_\-\_\-dev\_\-\_\-printConsole}}
\index{\_\-\_\-dev\_\-\_\-printConsole@{\_\-\_\-dev\_\-\_\-printConsole}!PVirtualMachine@{PVirtualMachine}}
\subsubsection[{\_\-\_\-dev\_\-\_\-printConsole}]{\setlength{\rightskip}{0pt plus 5cm}void PVirtualMachine::\_\-\_\-dev\_\-\_\-printConsole ()}}
\label{classPVirtualMachine_9fbb7c0313e22051eebfd2e316ec609b}


METODA TESTOWA. Wyświetla informacje o całej maszynie i jej elementach składowych. \hypertarget{classPVirtualMachine_d0ac4994cbf5c218a40442df49a3cd29}{
\index{PVirtualMachine@{PVirtualMachine}!\_\-\_\-dev\_\-\_\-printImageInfo@{\_\-\_\-dev\_\-\_\-printImageInfo}}
\index{\_\-\_\-dev\_\-\_\-printImageInfo@{\_\-\_\-dev\_\-\_\-printImageInfo}!PVirtualMachine@{PVirtualMachine}}
\subsubsection[{\_\-\_\-dev\_\-\_\-printImageInfo}]{\setlength{\rightskip}{0pt plus 5cm}void PVirtualMachine::\_\-\_\-dev\_\-\_\-printImageInfo ()}}
\label{classPVirtualMachine_d0ac4994cbf5c218a40442df49a3cd29}


METODA TESTOWA. Wyświetla informacje o obrazie kodu. \hypertarget{classPVirtualMachine_91c888349242ed7700d5ddff5b187c19}{
\index{PVirtualMachine@{PVirtualMachine}!\_\-\_\-dev\_\-\_\-printInstruction@{\_\-\_\-dev\_\-\_\-printInstruction}}
\index{\_\-\_\-dev\_\-\_\-printInstruction@{\_\-\_\-dev\_\-\_\-printInstruction}!PVirtualMachine@{PVirtualMachine}}
\subsubsection[{\_\-\_\-dev\_\-\_\-printInstruction}]{\setlength{\rightskip}{0pt plus 5cm}void PVirtualMachine::\_\-\_\-dev\_\-\_\-printInstruction ({\bf PInstructions} {\em instr})}}
\label{classPVirtualMachine_91c888349242ed7700d5ddff5b187c19}


METODA TESTOWA. Wyświetla nazwę zadanej instrukcji Pieta. \begin{Desc}
\item[Parametry:]
\begin{description}
\item[{\em instr}]instrukcja \end{description}
\end{Desc}
\hypertarget{classPVirtualMachine_40e1780e122ce7fe4ebadc526c47fbea}{
\index{PVirtualMachine@{PVirtualMachine}!executeAllInstr@{executeAllInstr}}
\index{executeAllInstr@{executeAllInstr}!PVirtualMachine@{PVirtualMachine}}
\subsubsection[{executeAllInstr}]{\setlength{\rightskip}{0pt plus 5cm}void PVirtualMachine::executeAllInstr ()}}
\label{classPVirtualMachine_40e1780e122ce7fe4ebadc526c47fbea}


Wykonuje WSZYSTKIE instrukcje aż do zakończenia pracy programu. Jeśli tryb gadatliwy został uprzednio włączony, wszystkie informacje o przebeigu pracy są wyświetlane. \hypertarget{classPVirtualMachine_7bbb22aabe782de2208139bc01f65050}{
\index{PVirtualMachine@{PVirtualMachine}!executeSingleInstr@{executeSingleInstr}}
\index{executeSingleInstr@{executeSingleInstr}!PVirtualMachine@{PVirtualMachine}}
\subsubsection[{executeSingleInstr}]{\setlength{\rightskip}{0pt plus 5cm}bool PVirtualMachine::executeSingleInstr ()}}
\label{classPVirtualMachine_7bbb22aabe782de2208139bc01f65050}


Wykonuje POJEDYNCZĄ instrukcję aż do zakończenia pracy programu. Jeśli tryb gadatliwy został uprzednio włączony, wszystkie informacje o przebeigu pracy są wyświetlane. \hypertarget{classPVirtualMachine_69d12176d3f9719b2bdf93a9f55cefde}{
\index{PVirtualMachine@{PVirtualMachine}!getInstructionByIndex@{getInstructionByIndex}}
\index{getInstructionByIndex@{getInstructionByIndex}!PVirtualMachine@{PVirtualMachine}}
\subsubsection[{getInstructionByIndex}]{\setlength{\rightskip}{0pt plus 5cm}{\bf PInstructions} PVirtualMachine::getInstructionByIndex (int {\em index})}}
\label{classPVirtualMachine_69d12176d3f9719b2bdf93a9f55cefde}


Zwraca instrukcję Pieta (element enumeracji) dla zadanego indeksu instrukcji Pieta. \begin{Desc}
\item[Parametry:]
\begin{description}
\item[{\em index}]indeks instrukcji Pieta \end{description}
\end{Desc}
\begin{Desc}
\item[Zwraca:]instrukcja Pieta \end{Desc}
\hypertarget{classPVirtualMachine_699839f4df2106065458127c36b4bfca}{
\index{PVirtualMachine@{PVirtualMachine}!getState@{getState}}
\index{getState@{getState}!PVirtualMachine@{PVirtualMachine}}
\subsubsection[{getState}]{\setlength{\rightskip}{0pt plus 5cm}{\bf PMachineStates} PVirtualMachine::getState ()\hspace{0.3cm}{\tt  \mbox{[}protected\mbox{]}}}}
\label{classPVirtualMachine_699839f4df2106065458127c36b4bfca}


zwraca stan wirtualnej maszyny Salvadora 

W każdej chwili istnienia, wirtualna maszyna Pieta musi znajdować się w jakimś stanie (PMachineStates). Metoda zwraca aktualny stan wirtualnej maszyny. \hypertarget{classPVirtualMachine_6b19e3e51a584b13e79c44fc5b842c2b}{
\index{PVirtualMachine@{PVirtualMachine}!isFinished@{isFinished}}
\index{isFinished@{isFinished}!PVirtualMachine@{PVirtualMachine}}
\subsubsection[{isFinished}]{\setlength{\rightskip}{0pt plus 5cm}bool PVirtualMachine::isFinished ()}}
\label{classPVirtualMachine_6b19e3e51a584b13e79c44fc5b842c2b}


Sprawdza czy maszyna zakończyła działanie (sprawdzany stan maszyny). \begin{Desc}
\item[Zwraca:]czy maszyna zakończyła działanie \end{Desc}
\hypertarget{classPVirtualMachine_5132bedd5e8480185b48ea6c4783c770}{
\index{PVirtualMachine@{PVirtualMachine}!isReady@{isReady}}
\index{isReady@{isReady}!PVirtualMachine@{PVirtualMachine}}
\subsubsection[{isReady}]{\setlength{\rightskip}{0pt plus 5cm}bool PVirtualMachine::isReady ()}}
\label{classPVirtualMachine_5132bedd5e8480185b48ea6c4783c770}


Sprawdza czy maszyna jest gotowa do rozpoczęcia pracy (sprawdzany stan maszyny). \begin{Desc}
\item[Zwraca:]czy maszyna jest gotowa do uruchomienia \end{Desc}
\hypertarget{classPVirtualMachine_823195d3ae1615c791dfeb6c8d365a0e}{
\index{PVirtualMachine@{PVirtualMachine}!isRunning@{isRunning}}
\index{isRunning@{isRunning}!PVirtualMachine@{PVirtualMachine}}
\subsubsection[{isRunning}]{\setlength{\rightskip}{0pt plus 5cm}bool PVirtualMachine::isRunning ()}}
\label{classPVirtualMachine_823195d3ae1615c791dfeb6c8d365a0e}


Sprawdza czy maszyna pracuje (sprawdzany stan maszyny). \begin{Desc}
\item[Zwraca:]czy maszyna pracuje \end{Desc}
\hypertarget{classPVirtualMachine_219cd1d08de6d3d5a9117eecc5348d9c}{
\index{PVirtualMachine@{PVirtualMachine}!isVerbose@{isVerbose}}
\index{isVerbose@{isVerbose}!PVirtualMachine@{PVirtualMachine}}
\subsubsection[{isVerbose}]{\setlength{\rightskip}{0pt plus 5cm}bool PVirtualMachine::isVerbose ()}}
\label{classPVirtualMachine_219cd1d08de6d3d5a9117eecc5348d9c}


Sprawdza, czy maszyna ma włączony tryb gadatliwy \begin{Desc}
\item[Zwraca:]tryb gadatliwy \end{Desc}
\hypertarget{classPVirtualMachine_29279548406a588bce03e320354cd857}{
\index{PVirtualMachine@{PVirtualMachine}!movePointerAndGetInstructionToExecute@{movePointerAndGetInstructionToExecute}}
\index{movePointerAndGetInstructionToExecute@{movePointerAndGetInstructionToExecute}!PVirtualMachine@{PVirtualMachine}}
\subsubsection[{movePointerAndGetInstructionToExecute}]{\setlength{\rightskip}{0pt plus 5cm}{\bf PInstructions} PVirtualMachine::movePointerAndGetInstructionToExecute ()}}
\label{classPVirtualMachine_29279548406a588bce03e320354cd857}


Metoda nakazuje przesunięcie głowicy oraz wyznacza jaka instrukcja ma zostać wykonana (wszystko wykorzystując pomocnicze obiekty wirtualnej maszyny). Jedna z ważniejszych i bardziej skomplikowanych metod całego interpretera. \hypertarget{classPVirtualMachine_6daa9fea05af4bfc30b881c893fbd725}{
\index{PVirtualMachine@{PVirtualMachine}!pointIsBlackOrOutside@{pointIsBlackOrOutside}}
\index{pointIsBlackOrOutside@{pointIsBlackOrOutside}!PVirtualMachine@{PVirtualMachine}}
\subsubsection[{pointIsBlackOrOutside}]{\setlength{\rightskip}{0pt plus 5cm}bool PVirtualMachine::pointIsBlackOrOutside ({\bf PPoint} {\em point})}}
\label{classPVirtualMachine_6daa9fea05af4bfc30b881c893fbd725}


Sprawdza, czy wskazany punkt jest koloru czarnego lub się znajduje poza granicami obrazu kodu. Jeśli tak, głowica będzie musiała zmienić swoje DP i/lub CC. Wirtualna maszyna wykorzystuje do tego głowicę obrazu kodu oraz menadżera kolorów. \begin{Desc}
\item[Parametry:]
\begin{description}
\item[{\em point}]wskazany punkt \end{description}
\end{Desc}
\begin{Desc}
\item[Zwraca:]czy punkt jest czarny lub poza granicami obrazu kodu \end{Desc}
\hypertarget{classPVirtualMachine_cae537f70c35487220818e8d18978e07}{
\index{PVirtualMachine@{PVirtualMachine}!pointIsWhite@{pointIsWhite}}
\index{pointIsWhite@{pointIsWhite}!PVirtualMachine@{PVirtualMachine}}
\subsubsection[{pointIsWhite}]{\setlength{\rightskip}{0pt plus 5cm}bool PVirtualMachine::pointIsWhite ({\bf PPoint} {\em point})}}
\label{classPVirtualMachine_cae537f70c35487220818e8d18978e07}


Sprawdza, czy wskazany punkt jest koloru białego. Jeśli tak, to jeśli głowica dodatkowo prześlizgnie się przez biały blok, nie wykona w tym krokużadnej operacji. Wirtualna maszyna wykorzystuje do tego głowicę obrazu kodu oraz menadżera kolorów. \begin{Desc}
\item[Parametry:]
\begin{description}
\item[{\em point}]wskazany punkt \end{description}
\end{Desc}
\begin{Desc}
\item[Zwraca:]czy punkt jest biały \end{Desc}
\hypertarget{classPVirtualMachine_513af8673f1430cd04fd0a0e46abedd4}{
\index{PVirtualMachine@{PVirtualMachine}!prepareToExecute@{prepareToExecute}}
\index{prepareToExecute@{prepareToExecute}!PVirtualMachine@{PVirtualMachine}}
\subsubsection[{prepareToExecute}]{\setlength{\rightskip}{0pt plus 5cm}void PVirtualMachine::prepareToExecute ()\hspace{0.3cm}{\tt  \mbox{[}protected\mbox{]}}}}
\label{classPVirtualMachine_513af8673f1430cd04fd0a0e46abedd4}


Przygotowanie do uruchomienia/zresetowania wirtualnej maszyny. 

Metoda używana w celu przygotowania wirtualnej maszyny do uruchomienia. Maszyna mogła być w trakcie działania, mogła zakończyć działanie lub mogła być gotowa do uruchomienia. \hypertarget{classPVirtualMachine_f4cdf5d80b7a169e2dfdebffe2588e4b}{
\index{PVirtualMachine@{PVirtualMachine}!restartMachine@{restartMachine}}
\index{restartMachine@{restartMachine}!PVirtualMachine@{PVirtualMachine}}
\subsubsection[{restartMachine}]{\setlength{\rightskip}{0pt plus 5cm}bool PVirtualMachine::restartMachine ()}}
\label{classPVirtualMachine_f4cdf5d80b7a169e2dfdebffe2588e4b}


Restartuje maszynę (ustawia odpowiedni stan). \begin{Desc}
\item[Zwraca:]czy operacja się powiodła \end{Desc}
\hypertarget{classPVirtualMachine_bede8c1e068c19b9bc61a957de70cda8}{
\index{PVirtualMachine@{PVirtualMachine}!setState@{setState}}
\index{setState@{setState}!PVirtualMachine@{PVirtualMachine}}
\subsubsection[{setState}]{\setlength{\rightskip}{0pt plus 5cm}void PVirtualMachine::setState ({\bf PMachineStates} {\em state})\hspace{0.3cm}{\tt  \mbox{[}protected\mbox{]}}}}
\label{classPVirtualMachine_bede8c1e068c19b9bc61a957de70cda8}


ustawia stan wirtualnej maszyny Salvadora 

W każdej chwili istnienia, wirtualna maszyna Pieta musi znajdować się w jakimś stanie (PMachineStates). Metoda ustawia stan wirtualnej maszyny. \begin{Desc}
\item[Parametry:]
\begin{description}
\item[{\em state}]stan jaki zostanie przypisany wirtualnej maszynie \end{description}
\end{Desc}
\hypertarget{classPVirtualMachine_4b2ebb998393324c5240dc48fdb14f3a}{
\index{PVirtualMachine@{PVirtualMachine}!slideAcrossWhiteBlock@{slideAcrossWhiteBlock}}
\index{slideAcrossWhiteBlock@{slideAcrossWhiteBlock}!PVirtualMachine@{PVirtualMachine}}
\subsubsection[{slideAcrossWhiteBlock}]{\setlength{\rightskip}{0pt plus 5cm}void PVirtualMachine::slideAcrossWhiteBlock ({\bf PPoint} \& {\em point})}}
\label{classPVirtualMachine_4b2ebb998393324c5240dc48fdb14f3a}


Przesuwa głowicę, uwzględniając jej wartości DP i CC, na koniec białego bloku (parametr point, przekazywany przez zmienną - ulega zmianie w trakcie wykonywania operacji). Następnie wirtualna maszyna bada, czy otrzymany punkt znajduje się poza obrazem kodu lub jest czarny (wtedy głowica zmienia wartości DP i/lub CC i być może ponownie będzie wędrowała z wyjściowego punktu przez biały blok). \begin{Desc}
\item[Parametry:]
\begin{description}
\item[{\em point}]wskazany punkt \end{description}
\end{Desc}
\hypertarget{classPVirtualMachine_e37d4dfa512c3dd97b9958f013ffb7b3}{
\index{PVirtualMachine@{PVirtualMachine}!slidePointerAcrossWhiteBlock@{slidePointerAcrossWhiteBlock}}
\index{slidePointerAcrossWhiteBlock@{slidePointerAcrossWhiteBlock}!PVirtualMachine@{PVirtualMachine}}
\subsubsection[{slidePointerAcrossWhiteBlock}]{\setlength{\rightskip}{0pt plus 5cm}void PVirtualMachine::slidePointerAcrossWhiteBlock ()}}
\label{classPVirtualMachine_e37d4dfa512c3dd97b9958f013ffb7b3}


Przesuwa głowicę o jeden kodel przez biały blok. Metoda wywoływana wielokrotnie przez metodę \hyperlink{classPVirtualMachine_4b2ebb998393324c5240dc48fdb14f3a}{slideAcrossWhiteBlock()}. \hypertarget{classPVirtualMachine_4b6bb128d63569e855684128b5e09f1c}{
\index{PVirtualMachine@{PVirtualMachine}!startMachine@{startMachine}}
\index{startMachine@{startMachine}!PVirtualMachine@{PVirtualMachine}}
\subsubsection[{startMachine}]{\setlength{\rightskip}{0pt plus 5cm}bool PVirtualMachine::startMachine ()}}
\label{classPVirtualMachine_4b6bb128d63569e855684128b5e09f1c}


Uruchamia maszynę (ustawia odpowiedni stan). \begin{Desc}
\item[Zwraca:]czy operacja się powiodła \end{Desc}
\hypertarget{classPVirtualMachine_fa30f1871832c1af18519db03c397ac0}{
\index{PVirtualMachine@{PVirtualMachine}!stopMachine@{stopMachine}}
\index{stopMachine@{stopMachine}!PVirtualMachine@{PVirtualMachine}}
\subsubsection[{stopMachine}]{\setlength{\rightskip}{0pt plus 5cm}bool PVirtualMachine::stopMachine ()}}
\label{classPVirtualMachine_fa30f1871832c1af18519db03c397ac0}


Zatrzymuje maszynę (ustawia odpowiedni stan). \begin{Desc}
\item[Zwraca:]czy operacja się powiodła \end{Desc}
\hypertarget{classPVirtualMachine_1e48dcf68b3b80d7f6d3c67b3f796e51}{
\index{PVirtualMachine@{PVirtualMachine}!toggleVerbosity@{toggleVerbosity}}
\index{toggleVerbosity@{toggleVerbosity}!PVirtualMachine@{PVirtualMachine}}
\subsubsection[{toggleVerbosity}]{\setlength{\rightskip}{0pt plus 5cm}void PVirtualMachine::toggleVerbosity ()}}
\label{classPVirtualMachine_1e48dcf68b3b80d7f6d3c67b3f796e51}


Przełącza tryb gadatliwy na przeciwny. Przełącza tryb we wszystkich podrzędnych obiektach. 

\subsection{Dokumentacja atrybutów składowych}
\hypertarget{classPVirtualMachine_67feeb2d88b4c8b766d10d6f7d8eb65b}{
\index{PVirtualMachine@{PVirtualMachine}!console@{console}}
\index{console@{console}!PVirtualMachine@{PVirtualMachine}}
\subsubsection[{console}]{\setlength{\rightskip}{0pt plus 5cm}{\bf PConsole}$\ast$ {\bf PVirtualMachine::console}\hspace{0.3cm}{\tt  \mbox{[}protected\mbox{]}}}}
\label{classPVirtualMachine_67feeb2d88b4c8b766d10d6f7d8eb65b}


Konsola wejścia/wyjścia Pieta, obsługująca operacje komunikacji z użytkownikiem \hypertarget{classPVirtualMachine_fe929445de96589087a96a8f1188fac0}{
\index{PVirtualMachine@{PVirtualMachine}!stack@{stack}}
\index{stack@{stack}!PVirtualMachine@{PVirtualMachine}}
\subsubsection[{stack}]{\setlength{\rightskip}{0pt plus 5cm}{\bf PCalcStack}$\ast$ {\bf PVirtualMachine::stack}\hspace{0.3cm}{\tt  \mbox{[}protected\mbox{]}}}}
\label{classPVirtualMachine_fe929445de96589087a96a8f1188fac0}


Stos przechowujący wszystkie tymczasowe wartości 

Dokumentacja dla tej klasy została wygenerowana z plików:\begin{CompactItemize}
\item 
src/core/\hyperlink{pvirtualmachine_8h}{pvirtualmachine.h}\item 
src/core/\hyperlink{pvirtualmachine_8cpp}{pvirtualmachine.cpp}\end{CompactItemize}

\chapter{Dokumentacja plików}
\hypertarget{pblockmanager_8cpp}{
\section{Dokumentacja pliku src/core/pblockmanager.cpp}
\label{pblockmanager_8cpp}\index{src/core/pblockmanager.cpp@{src/core/pblockmanager.cpp}}
}
Plik z kodem źródłowym klasy \hyperlink{classPBlockManager}{PBlockManager}.  


{\tt \#include \char`\"{}pblockmanager.h\char`\"{}}\par
{\tt \#include \char`\"{}../debug.h\char`\"{}}\par
{\tt \#include \char`\"{}penums.h\char`\"{}}\par
{\tt \#include \char`\"{}pstructs.h\char`\"{}}\par
{\tt \#include \char`\"{}pcodepointer.h\char`\"{}}\par
{\tt \#include $<$iostream$>$}\par
{\tt \#include $<$QImage$>$}\par
{\tt \#include $<$QRgb$>$}\par


\subsection{Opis szczegółowy}
Plik z kodem źródłowym klasy \hyperlink{classPBlockManager}{PBlockManager}. 

Plik zawiera kod źródłowy klasy \hyperlink{classPBlockManager}{PBlockManager}. 
\hypertarget{pblockmanager_8h}{
\section{Dokumentacja pliku src/core/pblockmanager.h}
\label{pblockmanager_8h}\index{src/core/pblockmanager.h@{src/core/pblockmanager.h}}
}
plik nagłówkowy klasy \hyperlink{classPBlockManager}{PBlockManager}  


{\tt \#include \char`\"{}penums.h\char`\"{}}\par
{\tt \#include \char`\"{}pstructs.h\char`\"{}}\par
{\tt \#include \char`\"{}pcodepointer.h\char`\"{}}\par
{\tt \#include $<$QRgb$>$}\par
{\tt \#include $<$QImage$>$}\par
\subsection*{Komponenty}
\begin{CompactItemize}
\item 
class \hyperlink{classPBlockManager}{PBlockManager}
\begin{CompactList}\small\item\em Manager bloków kolorów. \item\end{CompactList}\end{CompactItemize}


\subsection{Opis szczegółowy}
plik nagłówkowy klasy \hyperlink{classPBlockManager}{PBlockManager} 

Plik nagłówkowy zawiera definicję klasy \hyperlink{classPBlockManager}{PBlockManager}. 
\hypertarget{pcalcstack_8cpp}{
\section{Dokumentacja pliku src/core/pcalcstack.cpp}
\label{pcalcstack_8cpp}\index{src/core/pcalcstack.cpp@{src/core/pcalcstack.cpp}}
}
Plik z kodem źródłowym klasy \hyperlink{classPCalcStack}{PCalcStack}.  


{\tt \#include \char`\"{}pcalcstack.h\char`\"{}}\par
{\tt \#include \char`\"{}../debug.h\char`\"{}}\par
{\tt \#include \char`\"{}penums.h\char`\"{}}\par
{\tt \#include \char`\"{}pstructs.h\char`\"{}}\par
{\tt \#include $<$iostream$>$}\par
{\tt \#include $<$list$>$}\par


\subsection{Opis szczegółowy}
Plik z kodem źródłowym klasy \hyperlink{classPCalcStack}{PCalcStack}. 

Plik zawiera kod źródłowy klasy \hyperlink{classPCalcStack}{PCalcStack}. 
\hypertarget{pcalcstack_8h}{
\section{Dokumentacja pliku src/core/pcalcstack.h}
\label{pcalcstack_8h}\index{src/core/pcalcstack.h@{src/core/pcalcstack.h}}
}
plik nagłówkowy klasy \hyperlink{classPCalcStack}{PCalcStack}  


{\tt \#include \char`\"{}penums.h\char`\"{}}\par
{\tt \#include \char`\"{}pstructs.h\char`\"{}}\par
{\tt \#include $<$list$>$}\par
\subsection*{Komponenty}
\begin{CompactItemize}
\item 
class \hyperlink{classPCalcStack}{PCalcStack}
\begin{CompactList}\small\item\em Stos. \item\end{CompactList}\end{CompactItemize}


\subsection{Opis szczegółowy}
plik nagłówkowy klasy \hyperlink{classPCalcStack}{PCalcStack} 

Plik nagłówkowy zawiera definicję klasy \hyperlink{classPCalcStack}{PCalcStack}. 
\hypertarget{pcodepointer_8cpp}{
\section{Dokumentacja pliku src/core/pcodepointer.cpp}
\label{pcodepointer_8cpp}\index{src/core/pcodepointer.cpp@{src/core/pcodepointer.cpp}}
}
Plik z kodem źródłowym klasy \hyperlink{classPCodePointer}{PCodePointer}.  


{\tt \#include \char`\"{}pcodepointer.h\char`\"{}}\par
{\tt \#include \char`\"{}../debug.h\char`\"{}}\par
{\tt \#include \char`\"{}penums.h\char`\"{}}\par
{\tt \#include \char`\"{}pstructs.h\char`\"{}}\par
{\tt \#include $<$iostream$>$}\par


\subsection{Opis szczegółowy}
Plik z kodem źródłowym klasy \hyperlink{classPCodePointer}{PCodePointer}. 

Plik zawiera kod źródłowy klasy \hyperlink{classPCodePointer}{PCodePointer}. 
\hypertarget{pcodepointer_8h}{
\section{Dokumentacja pliku src/core/pcodepointer.h}
\label{pcodepointer_8h}\index{src/core/pcodepointer.h@{src/core/pcodepointer.h}}
}
plik nagłówkowy klasy \hyperlink{classPCodePointer}{PCodePointer}  


{\tt \#include \char`\"{}penums.h\char`\"{}}\par
{\tt \#include \char`\"{}pstructs.h\char`\"{}}\par
{\tt \#include $<$QImage$>$}\par
{\tt \#include $<$QRgb$>$}\par
\subsection*{Komponenty}
\begin{CompactItemize}
\item 
class \hyperlink{classPCodePointer}{PCodePointer}
\begin{CompactList}\small\item\em Głowica obrazu kodu. \item\end{CompactList}\end{CompactItemize}


\subsection{Opis szczegółowy}
plik nagłówkowy klasy \hyperlink{classPCodePointer}{PCodePointer} 

Plik nagłówkowy zawiera definicję klasy \hyperlink{classPCodePointer}{PCodePointer}. 
\hypertarget{pcolormanager_8cpp}{
\section{Dokumentacja pliku src/core/pcolormanager.cpp}
\label{pcolormanager_8cpp}\index{src/core/pcolormanager.cpp@{src/core/pcolormanager.cpp}}
}
Plik z kodem źródłowym klasy \hyperlink{classPColorManager}{PColorManager}.  


{\tt \#include \char`\"{}pcolormanager.h\char`\"{}}\par
{\tt \#include \char`\"{}../debug.h\char`\"{}}\par
{\tt \#include \char`\"{}penums.h\char`\"{}}\par
{\tt \#include \char`\"{}pstructs.h\char`\"{}}\par
{\tt \#include $<$cstdlib$>$}\par
{\tt \#include $<$iostream$>$}\par
{\tt \#include $<$QRgb$>$}\par


\subsection{Opis szczegółowy}
Plik z kodem źródłowym klasy \hyperlink{classPColorManager}{PColorManager}. 

Plik zawiera kod źródłowy klasy \hyperlink{classPColorManager}{PColorManager}. 
\hypertarget{pcolormanager_8h}{
\section{Dokumentacja pliku src/core/pcolormanager.h}
\label{pcolormanager_8h}\index{src/core/pcolormanager.h@{src/core/pcolormanager.h}}
}
plik nagłówkowy klasy \hyperlink{classPColorManager}{PColorManager}  


{\tt \#include \char`\"{}penums.h\char`\"{}}\par
{\tt \#include \char`\"{}pstructs.h\char`\"{}}\par
{\tt \#include $<$QRgb$>$}\par
\subsection*{Komponenty}
\begin{CompactItemize}
\item 
class \hyperlink{classPColorManager}{PColorManager}
\begin{CompactList}\small\item\em Maszyna kodu. \item\end{CompactList}\end{CompactItemize}


\subsection{Opis szczegółowy}
plik nagłówkowy klasy \hyperlink{classPColorManager}{PColorManager} 

Plik nagłówkowy zawiera definicję klasy \hyperlink{classPColorManager}{PColorManager}. 
\hypertarget{pconsole_8cpp}{
\section{Dokumentacja pliku src/core/pconsole.cpp}
\label{pconsole_8cpp}\index{src/core/pconsole.cpp@{src/core/pconsole.cpp}}
}
Plik z kodem źródłowym klasy \hyperlink{classPConsole}{PConsole}.  


{\tt \#include \char`\"{}pconsole.h\char`\"{}}\par
{\tt \#include \char`\"{}../debug.h\char`\"{}}\par
{\tt \#include \char`\"{}penums.h\char`\"{}}\par
{\tt \#include \char`\"{}pstructs.h\char`\"{}}\par
{\tt \#include $<$iostream$>$}\par
{\tt \#include $<$iomanip$>$}\par


\subsection{Opis szczegółowy}
Plik z kodem źródłowym klasy \hyperlink{classPConsole}{PConsole}. 

Plik zawiera kod źródłowy klasy \hyperlink{classPConsole}{PConsole}. 
\hypertarget{pconsole_8h}{
\section{Dokumentacja pliku src/core/pconsole.h}
\label{pconsole_8h}\index{src/core/pconsole.h@{src/core/pconsole.h}}
}
plik nagłówkowy klasy \hyperlink{classPConsole}{PConsole}  


{\tt \#include \char`\"{}penums.h\char`\"{}}\par
{\tt \#include \char`\"{}pstructs.h\char`\"{}}\par
\subsection*{Komponenty}
\begin{CompactItemize}
\item 
class \hyperlink{classPConsole}{PConsole}
\begin{CompactList}\small\item\em Konsola wejścia/wyjścia. \item\end{CompactList}\end{CompactItemize}


\subsection{Opis szczegółowy}
plik nagłówkowy klasy \hyperlink{classPConsole}{PConsole} 

Plik nagłówkowy zawiera definicję klasy \hyperlink{classPConsole}{PConsole}. 
\hypertarget{penums_8h}{
\section{Dokumentacja pliku src/core/penums.h}
\label{penums_8h}\index{src/core/penums.h@{src/core/penums.h}}
}
wszystkie definicje enumeracji  


\subsection*{Wyliczenia}
\begin{CompactItemize}
\item 
enum \hyperlink{penums_8h_4fb01e50a2da85245f7cea7856eca8ec}{PStdColors} \{ \par
\textbf{color\_\-light\_\-red}, 
\textbf{color\_\-normal\_\-red}, 
\textbf{color\_\-dark\_\-red}, 
\textbf{color\_\-light\_\-yellow}, 
\par
\textbf{color\_\-normal\_\-yellow}, 
\textbf{color\_\-dark\_\-yellow}, 
\textbf{color\_\-light\_\-green}, 
\textbf{color\_\-normal\_\-green}, 
\par
\textbf{color\_\-dark\_\-green}, 
\textbf{color\_\-light\_\-cyan}, 
\textbf{color\_\-normal\_\-cyan}, 
\textbf{color\_\-dark\_\-cyan}, 
\par
\textbf{color\_\-light\_\-blue}, 
\textbf{color\_\-normal\_\-blue}, 
\textbf{color\_\-dark\_\-blue}, 
\textbf{color\_\-light\_\-magenta}, 
\par
\textbf{color\_\-normal\_\-magenta}, 
\textbf{color\_\-dark\_\-magenta}, 
\textbf{color\_\-white}, 
\textbf{color\_\-black}
 \}
\begin{CompactList}\small\item\em standardowe kolory \item\end{CompactList}\item 
enum \hyperlink{penums_8h_680063dc1cb094b4c3fb8715c3270776}{PNonStdColorBehavior} \{ \par
\textbf{beh\_\-treat\_\-as\_\-white}, 
\textbf{beh\_\-treat\_\-as\_\-black}, 
\textbf{beh\_\-nearest\_\-upper}, 
\textbf{beh\_\-nearest\_\-lower}, 
\par
\textbf{beh\_\-nearest\_\-neighbour}
 \}
\begin{CompactList}\small\item\em interpretacja niestandardowych kolorów \item\end{CompactList}\item 
enum \hyperlink{penums_8h_59dc57d526e2ce263bdf851c0d4fef3e}{PCodelChooserValues} \{ \textbf{cc\_\-left}, 
\textbf{cc\_\-right}
 \}
\begin{CompactList}\small\item\em tryby codel chooser \item\end{CompactList}\item 
enum \hyperlink{penums_8h_6d3256570150238c718cbbb5f81c82df}{PDirectionPointerValues} \{ \textbf{dp\_\-right}, 
\textbf{dp\_\-down}, 
\textbf{dp\_\-left}, 
\textbf{dp\_\-up}
 \}
\begin{CompactList}\small\item\em tryby direction pointer \item\end{CompactList}\item 
enum \hyperlink{penums_8h_3b12ec7c990da3435462047165bc0c6d}{PMachineStates} \{ \textbf{state\_\-ready}, 
\textbf{state\_\-running}, 
\textbf{state\_\-finished}
 \}
\begin{CompactList}\small\item\em stany maszyny wirtualnej \item\end{CompactList}\item 
enum \hyperlink{penums_8h_797d2a71195500b970298fbf455ff83a}{PInstructions} \{ \par
\textbf{pietInstr\_\-special\_\-empty}, 
\textbf{pietInstr\_\-stack\_\-push}, 
\textbf{pietInstr\_\-stack\_\-pop}, 
\textbf{pietInstr\_\-arithm\_\-add}, 
\par
\textbf{pietInstr\_\-arithm\_\-subtract}, 
\textbf{pietInstr\_\-arithm\_\-multiply}, 
\textbf{pietInstr\_\-arithm\_\-divide}, 
\textbf{pietInstr\_\-arithm\_\-modulo}, 
\par
\textbf{pietInstr\_\-logic\_\-not}, 
\textbf{pietInstr\_\-logic\_\-greater}, 
\textbf{pietInstr\_\-runtime\_\-pointer}, 
\textbf{pietInstr\_\-runtime\_\-switch}, 
\par
\textbf{pietInstr\_\-stack\_\-duplicate}, 
\textbf{pietInstr\_\-stack\_\-roll}, 
\textbf{pietInstr\_\-io\_\-in\_\-number}, 
\textbf{pietInstr\_\-io\_\-in\_\-char}, 
\par
\textbf{pietInstr\_\-io\_\-out\_\-number}, 
\textbf{pietInstr\_\-io\_\-out\_\-char}, 
\textbf{pietInstr\_\-special\_\-terminate}
 \}
\begin{CompactList}\small\item\em instrukcje Pieta \item\end{CompactList}\end{CompactItemize}


\subsection{Opis szczegółowy}
wszystkie definicje enumeracji 

Plik nagłówkowy zawiera wszystkie definicje enumeracji które są wykorzystywane w projekcie. 

\subsection{Dokumentacja typów wyliczanych}
\hypertarget{penums_8h_59dc57d526e2ce263bdf851c0d4fef3e}{
\index{penums.h@{penums.h}!PCodelChooserValues@{PCodelChooserValues}}
\index{PCodelChooserValues@{PCodelChooserValues}!penums.h@{penums.h}}
\subsubsection[{PCodelChooserValues}]{\setlength{\rightskip}{0pt plus 5cm}enum {\bf PCodelChooserValues}}}
\label{penums_8h_59dc57d526e2ce263bdf851c0d4fef3e}


tryby codel chooser 

Enumeracja wyszczególniająca tryby pracy codel chooser, czyli kierunki w których głowica ma szukać skrajnego kodela po dojściu do granicy bloku w kierunku wskazanym przez direction pointer. \hypertarget{penums_8h_6d3256570150238c718cbbb5f81c82df}{
\index{penums.h@{penums.h}!PDirectionPointerValues@{PDirectionPointerValues}}
\index{PDirectionPointerValues@{PDirectionPointerValues}!penums.h@{penums.h}}
\subsubsection[{PDirectionPointerValues}]{\setlength{\rightskip}{0pt plus 5cm}enum {\bf PDirectionPointerValues}}}
\label{penums_8h_6d3256570150238c718cbbb5f81c82df}


tryby direction pointer 

Enumeracja wyszczególniająca tryby pracy direction pointer, czyli kierunki w których głowica ma poruszać się aż dojdzie do granicy bloku kolorów (potem codel chooser określa z której strony przejdzie do nowego bloku - dzięki temu glowica przechodzi z jednego bloku kolorów do drugiego i wykonują się instrukcje Pieta). \hypertarget{penums_8h_797d2a71195500b970298fbf455ff83a}{
\index{penums.h@{penums.h}!PInstructions@{PInstructions}}
\index{PInstructions@{PInstructions}!penums.h@{penums.h}}
\subsubsection[{PInstructions}]{\setlength{\rightskip}{0pt plus 5cm}enum {\bf PInstructions}}}
\label{penums_8h_797d2a71195500b970298fbf455ff83a}


instrukcje Pieta 

Enumeracja wyszczególniająca wszystkie instrukcje wykonywane przez język Piet. \hypertarget{penums_8h_3b12ec7c990da3435462047165bc0c6d}{
\index{penums.h@{penums.h}!PMachineStates@{PMachineStates}}
\index{PMachineStates@{PMachineStates}!penums.h@{penums.h}}
\subsubsection[{PMachineStates}]{\setlength{\rightskip}{0pt plus 5cm}enum {\bf PMachineStates}}}
\label{penums_8h_3b12ec7c990da3435462047165bc0c6d}


stany maszyny wirtualnej 

Enumeracja wyszczególniająca stany tzw. \char`\"{}maszyny wirtualnej Pieta\char`\"{}, która zajmuje się pełną interpretacją kodu Pieta. \hypertarget{penums_8h_680063dc1cb094b4c3fb8715c3270776}{
\index{penums.h@{penums.h}!PNonStdColorBehavior@{PNonStdColorBehavior}}
\index{PNonStdColorBehavior@{PNonStdColorBehavior}!penums.h@{penums.h}}
\subsubsection[{PNonStdColorBehavior}]{\setlength{\rightskip}{0pt plus 5cm}enum {\bf PNonStdColorBehavior}}}
\label{penums_8h_680063dc1cb094b4c3fb8715c3270776}


interpretacja niestandardowych kolorów 

Enumeracja wyszczególniająca możliwe sposoby interpretacji kolorów niestandardowych \hypertarget{penums_8h_4fb01e50a2da85245f7cea7856eca8ec}{
\index{penums.h@{penums.h}!PStdColors@{PStdColors}}
\index{PStdColors@{PStdColors}!penums.h@{penums.h}}
\subsubsection[{PStdColors}]{\setlength{\rightskip}{0pt plus 5cm}enum {\bf PStdColors}}}
\label{penums_8h_4fb01e50a2da85245f7cea7856eca8ec}


standardowe kolory 

Enumeracja wyszczególniająca kolory standardowe wykorzystywane do określania wykonywanej instrukcji 
\hypertarget{pstructs_8h}{
\section{Dokumentacja pliku src/core/pstructs.h}
\label{pstructs_8h}\index{src/core/pstructs.h@{src/core/pstructs.h}}
}
wszystkie definicje rekordów  


\subsection*{Komponenty}
\begin{CompactItemize}
\item 
struct \textbf{struct\_\-point}
\end{CompactItemize}
\subsection*{Definicje typów}
\begin{CompactItemize}
\item 
\hypertarget{pstructs_8h_8a0d654b5ccaaa40a0208f05a6225d4f}{
typedef struct struct\_\-point \textbf{PPoint}}
\label{pstructs_8h_8a0d654b5ccaaa40a0208f05a6225d4f}

\end{CompactItemize}


\subsection{Opis szczegółowy}
wszystkie definicje rekordów 

Plik nagłówkowy zawiera wszystkie definicje rekordów które są wykorzystywane w projekcie. 
\hypertarget{pvirtualmachine_8cpp}{
\section{Dokumentacja pliku src/core/pvirtualmachine.cpp}
\label{pvirtualmachine_8cpp}\index{src/core/pvirtualmachine.cpp@{src/core/pvirtualmachine.cpp}}
}
Plik z kodem źródłowym klasy \hyperlink{classPVirtualMachine}{PVirtualMachine}.  


{\tt \#include \char`\"{}pvirtualmachine.h\char`\"{}}\par
{\tt \#include \char`\"{}../debug.h\char`\"{}}\par
{\tt \#include \char`\"{}penums.h\char`\"{}}\par
{\tt \#include \char`\"{}pstructs.h\char`\"{}}\par
{\tt \#include \char`\"{}pconsole.h\char`\"{}}\par
{\tt \#include \char`\"{}pcalcstack.h\char`\"{}}\par
{\tt \#include $<$iostream$>$}\par
{\tt \#include $<$QString$>$}\par
{\tt \#include $<$QImage$>$}\par
{\tt \#include $<$QRgb$>$}\par


\subsection{Opis szczegółowy}
Plik z kodem źródłowym klasy \hyperlink{classPVirtualMachine}{PVirtualMachine}. 

Plik zawiera kod źródłowy klasy \hyperlink{classPVirtualMachine}{PVirtualMachine}. 
\hypertarget{pvirtualmachine_8h}{
\section{Dokumentacja pliku src/core/pvirtualmachine.h}
\label{pvirtualmachine_8h}\index{src/core/pvirtualmachine.h@{src/core/pvirtualmachine.h}}
}
plik nagłówkowy klasy \hyperlink{classPVirtualMachine}{PVirtualMachine}  


{\tt \#include \char`\"{}penums.h\char`\"{}}\par
{\tt \#include \char`\"{}pstructs.h\char`\"{}}\par
{\tt \#include \char`\"{}pconsole.h\char`\"{}}\par
{\tt \#include \char`\"{}pblockmanager.h\char`\"{}}\par
{\tt \#include \char`\"{}pcolormanager.h\char`\"{}}\par
{\tt \#include \char`\"{}pcodepointer.h\char`\"{}}\par
{\tt \#include \char`\"{}pcalcstack.h\char`\"{}}\par
{\tt \#include $<$QString$>$}\par
{\tt \#include $<$QImage$>$}\par
\subsection*{Komponenty}
\begin{CompactItemize}
\item 
class \hyperlink{classPVirtualMachine}{PVirtualMachine}
\begin{CompactList}\small\item\em Wirtualna maszyna Pieta. \item\end{CompactList}\end{CompactItemize}


\subsection{Opis szczegółowy}
plik nagłówkowy klasy \hyperlink{classPVirtualMachine}{PVirtualMachine} 

Plik nagłówkowy zawiera definicję klasy \hyperlink{classPVirtualMachine}{PVirtualMachine}. 
\hypertarget{debug_8h}{
\section{Dokumentacja pliku src/debug.h}
\label{debug_8h}\index{src/debug.h@{src/debug.h}}
}
Plik nagłówkowy debuggera.  


\subsection*{Definicje}
\begin{CompactItemize}
\item 
\hypertarget{debug_8h_86ee3ff44c537d94ccbabf941a613688}{
\#define \textbf{debug}(...)}
\label{debug_8h_86ee3ff44c537d94ccbabf941a613688}

\end{CompactItemize}


\subsection{Opis szczegółowy}
Plik nagłówkowy debuggera. 

Plik zawiera makro definiujące operacje debuggera. 
\hypertarget{test_8cpp}{
\section{Dokumentacja pliku src/test.cpp}
\label{test_8cpp}\index{src/test.cpp@{src/test.cpp}}
}
Plik z kodem źródłowym aplikacji.  


{\tt \#include \char`\"{}core/pvirtualmachine.h\char`\"{}}\par
{\tt \#include \char`\"{}core/penums.h\char`\"{}}\par
{\tt \#include \char`\"{}debug.h\char`\"{}}\par
{\tt \#include $<$iostream$>$}\par
{\tt \#include $<$fstream$>$}\par
{\tt \#include $<$string$>$}\par
{\tt \#include $<$QString$>$}\par
\subsection*{Funkcje}
\begin{CompactItemize}
\item 
void \hyperlink{test_8cpp_3ebfca17648870bd0b802e3742db1b51}{setConsoleColor} (int color)
\item 
void \hyperlink{test_8cpp_d15ce8354f955da3aa71e04a89011c50}{printFormattedError} (std::string error)
\item 
void \hyperlink{test_8cpp_b771de3d51d2056379290ba1f7f981f2}{printFormattedMessage} (std::string message)
\item 
void \hyperlink{test_8cpp_4daa93846a5ef739539e10772d176bc1}{runWelcome} ()
\item 
int \hyperlink{test_8cpp_a01ef0daafd23a3cca3b70eb23e01151}{runMenu} ()
\item 
void \hyperlink{test_8cpp_ace6b288e5cbf8113bb8b224b95fc08f}{runProgram} ()
\item 
int \hyperlink{test_8cpp_3c04138a5bfe5d72780bb7e82a18e627}{main} (int argc, char $\ast$$\ast$argv)
\end{CompactItemize}
\subsection*{Zmienne}
\begin{CompactItemize}
\item 
\hyperlink{classPVirtualMachine}{PVirtualMachine} $\ast$ \hyperlink{test_8cpp_18860b292576393855fd861b9b5a2499}{m}
\end{CompactItemize}


\subsection{Opis szczegółowy}
Plik z kodem źródłowym aplikacji. 

Plik zawiera kod źródłowy aplikacji testującej interpreter języka Piet. Program przeznaczony do użytku pod konsolą systemu Unix. 

\subsection{Dokumentacja funkcji}
\hypertarget{test_8cpp_3c04138a5bfe5d72780bb7e82a18e627}{
\index{test.cpp@{test.cpp}!main@{main}}
\index{main@{main}!test.cpp@{test.cpp}}
\subsubsection[{main}]{\setlength{\rightskip}{0pt plus 5cm}int main (int {\em argc}, \/  char $\ast$$\ast$ {\em argv})}}
\label{test_8cpp_3c04138a5bfe5d72780bb7e82a18e627}


Procedura wejściowa aplikacji, pośrednicząca z linią poleceń (wczytuje nazwę programu z tablicy parametrów). Tworzy wszystkie potrzebne zmienne i wywołuje \hyperlink{test_8cpp_ace6b288e5cbf8113bb8b224b95fc08f}{runProgram()}. \begin{Desc}
\item[Parametry:]
\begin{description}
\item[{\em argc}]liczba parametrów pobranych z komendy uruchamiającej program \item[{\em argv}]tablica z wartościami parametrów pobranych z komendy uruchamiającej program \end{description}
\end{Desc}
\hypertarget{test_8cpp_d15ce8354f955da3aa71e04a89011c50}{
\index{test.cpp@{test.cpp}!printFormattedError@{printFormattedError}}
\index{printFormattedError@{printFormattedError}!test.cpp@{test.cpp}}
\subsubsection[{printFormattedError}]{\setlength{\rightskip}{0pt plus 5cm}void printFormattedError (std::string {\em error})}}
\label{test_8cpp_d15ce8354f955da3aa71e04a89011c50}


Wyświetla komunikat błędu na konsolę. \begin{Desc}
\item[Parametry:]
\begin{description}
\item[{\em error}]tekst komunikatu błędu \end{description}
\end{Desc}
\hypertarget{test_8cpp_b771de3d51d2056379290ba1f7f981f2}{
\index{test.cpp@{test.cpp}!printFormattedMessage@{printFormattedMessage}}
\index{printFormattedMessage@{printFormattedMessage}!test.cpp@{test.cpp}}
\subsubsection[{printFormattedMessage}]{\setlength{\rightskip}{0pt plus 5cm}void printFormattedMessage (std::string {\em message})}}
\label{test_8cpp_b771de3d51d2056379290ba1f7f981f2}


Wyświetla standardowy komunikat na konsolę. \begin{Desc}
\item[Parametry:]
\begin{description}
\item[{\em message}]tekst standardowego komunikatu \end{description}
\end{Desc}
\hypertarget{test_8cpp_a01ef0daafd23a3cca3b70eb23e01151}{
\index{test.cpp@{test.cpp}!runMenu@{runMenu}}
\index{runMenu@{runMenu}!test.cpp@{test.cpp}}
\subsubsection[{runMenu}]{\setlength{\rightskip}{0pt plus 5cm}int runMenu ()}}
\label{test_8cpp_a01ef0daafd23a3cca3b70eb23e01151}


Wyświetla menu programu, prosi użytkownika o wybór zadania i zwraca ów wybór. \begin{Desc}
\item[Zwraca:]numer zadania wybrany przez użytkownika \end{Desc}
\hypertarget{test_8cpp_ace6b288e5cbf8113bb8b224b95fc08f}{
\index{test.cpp@{test.cpp}!runProgram@{runProgram}}
\index{runProgram@{runProgram}!test.cpp@{test.cpp}}
\subsubsection[{runProgram}]{\setlength{\rightskip}{0pt plus 5cm}void runProgram ()}}
\label{test_8cpp_ace6b288e5cbf8113bb8b224b95fc08f}


Główna procedura całej aplikacji. Wyświetla przywitanie, potem w pętli pobiera od użytkownika numer zadania i wykonuje je. Działanie programu zależy od decyzji użytkownika. \hypertarget{test_8cpp_4daa93846a5ef739539e10772d176bc1}{
\index{test.cpp@{test.cpp}!runWelcome@{runWelcome}}
\index{runWelcome@{runWelcome}!test.cpp@{test.cpp}}
\subsubsection[{runWelcome}]{\setlength{\rightskip}{0pt plus 5cm}void runWelcome ()}}
\label{test_8cpp_4daa93846a5ef739539e10772d176bc1}


Wyświetla tekst powitalny programu. \hypertarget{test_8cpp_3ebfca17648870bd0b802e3742db1b51}{
\index{test.cpp@{test.cpp}!setConsoleColor@{setConsoleColor}}
\index{setConsoleColor@{setConsoleColor}!test.cpp@{test.cpp}}
\subsubsection[{setConsoleColor}]{\setlength{\rightskip}{0pt plus 5cm}void setConsoleColor (int {\em color})}}
\label{test_8cpp_3ebfca17648870bd0b802e3742db1b51}


Ustawia kolor czcionki konsoli Unix. \begin{Desc}
\item[Parametry:]
\begin{description}
\item[{\em color}]liczba definiująca kolor czcionki pod konsolą (escape string) \end{description}
\end{Desc}


\subsection{Dokumentacja zmiennych}
\hypertarget{test_8cpp_18860b292576393855fd861b9b5a2499}{
\index{test.cpp@{test.cpp}!m@{m}}
\index{m@{m}!test.cpp@{test.cpp}}
\subsubsection[{m}]{\setlength{\rightskip}{0pt plus 5cm}{\bf PVirtualMachine}$\ast$ {\bf m}}}
\label{test_8cpp_18860b292576393855fd861b9b5a2499}


Wirtualna maszyna Pieta - globalna zmienna 
\printindex
\end{document}
